\chapter{Introduction}

The ability to select materials for specific problems is crucial to the success of any design. Recently, engineers have begun to design materials from the bottom up (e.i.: atom by atom) to harness nanoscale phenomena with the chance of discovering applications for exciting new technologies or improving the efficiencies of current ones. Modeling has become increasingly used a guide into the diverse and expansive world of \textit{small}.

The problem of predicting material properties remains a challenge to the present day. The arsenal of tools, however, has become increasingly powerful with the steady advance of computational perfomance in accordance with Moore's Law. Indeed, computational techniques like density functional theory (DFT), quantum Monte Carlo (QMC), and classical molecular dynamics (MD) have transitioned to the realm of the desktop personal computer (albeit for simple systems).

%By numerically evaluating the equations of motions for the system's particles (Newton's 2nd law for atoms in classical MD, Schrodinger's equation for electrons in DFT or QMC), material properties like electrical or thermal conductivity can be determined through the statistical behaviour of the particles' phase coordinates (e.i.: position and momentum).
Recent work has demonstrated the ability of DFT \cite{broido1,PhysRevB.84.085204} to accurately predict thermal conductivity. The underlying principle of DFT is to recast the problem of solving the many-body Schrodinger equation into as an optimization problem, where minimizing the objective function yields the ground state energy. The accuracy in the thermal conductivity prediction stems from using the quantum mechanics of the electrons to determine the interatomic interactions.

Molecular dynamics (MD), on the other hand, numerically integrates Newton's 2$^{nd}$ Law using an empirical potential as input. Atoms are generally treated as point masses connected by non-linear springs. The use of this empirical potential reduces the computational cost by orders of magnitude relative to DFT at the expense of physical accuracy. The methods used in this work, in some form or another, can in theory use input from DFT to improve accuracy in predicting thermal conductivity and/or phonon properties. MD is used instead of DFT in this work in order to compare as many models as possible under computational limitations.

We are motivated by previous work which used DFT or MD to predict thermal conductivity in bulk systems like bulk silicon \cite{PhysRevB.84.085204}, bulk germanium \cite{broido2007intrinsic} and bulk silicon/germanium alloys \cite{garg2011role}. The objective of this thesis is to apply a subset of the available computational methods to modeling thermal transport to nanostructured systems, specifically systems with interfaces. Of particular interest is the ability to predict phonon properties (group velocities, specific heats and lifetimes) in these nanostructured systems. Phonon properties can then be used as input for Boltzmann Transport Equation (BTE) solving methods like the Lattice Boltzmann Method (LBM) \cite{escobar2006multi,smith2006lattice,nabovati2011lattice} or Monte Carlo methods \cite{mazumder2001monte, lacroix2005monte, peraud2011efficient} when the empirical Fourier's Law is no longer valid \cite{cahill2003nanoscale}. %the Atomistic Green's Functions method \cite{zhang2007atomistic,hopkins2009extracting,xu2008nonequilibrium,wang2006nonequilibrium},

The outline of the thesis is as follows. Chapter~\ref{CHP:background} reviews four methods of predicting thermal conductivity: the Green-Kubo (GK) method, the Direct Method (DM), Anharmonic Lattice Dynamics (ALD) and Normal Mode Decomposition (NMD). Chapter~\ref{CHP:Interface} discusses the challenges in examining phonon properties near an interface. Chapter~\ref{CHP:SL} presents the results of applying NMD, ALD and GK to superlattices. Potential avenues for future work is contained in the conclusion.

