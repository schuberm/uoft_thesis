\chapter{Introduction}

The intent of this introduction is to briefly present the methods used to study nanoscale thermal transport without delving into the details and outline the sections of the thesis.

The problem of predicting material properties remains to a challenge to the present day. However, the arsenal of tools has become increasingly powerful with the steady advance of computational perfomance in accordance with Moore's Law. Indeed, computational techniques like Density Functional Theory (DFT), Quantum Monte Carlo (QMC), and classical Molecular Dynamics (MD) have transitioned to the realm of the desktop PC (albeit for simple systems).

By directly evaluating the equations of motions for the system's particles (Newton's 2nd law for atoms in classical MD, Schrodinger's equation for electrons in DFT or QMC), macroscopic properties like electrical or thermal conductivity can be abstracted through the statistical behaviour of the particles' phase coordinates (e.i.: position and momentum).

Of the multitude of computational techniques, those not discussed in this work but should be acknowledged are the Lattice Boltzmann Method (LBM), \cite{escobar2006multi,smith2006lattice,nabovati2011lattice} Atomistic Green's Functions method, \cite{zhang2007atomistic,hopkins2009extracting,xu2008nonequilibrium,wang2006nonequilibrium}
Density Functional Theory methods, \cite{broido1,PhysRevB.84.085204}
and Monte Carlo methods.\cite{mazumder2001monte, lacroix2005monte, peraud2011efficient}

The outline of the thesis is as follows: Chapter 1 reviews four methods of predicting thermal conductivity: Green-Kubo (GK), Direct Method (DM), Anharmonic Lattice Dynamics (ALD) and Normal Mode Decomposition (NMD). Chapter 2 discusses the challenges in examining phonon properties near an interface. Chapter 3 presents the results of a superlattice study. An outline of future work is contained in the conclusion.

