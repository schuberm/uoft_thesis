\chapter{Insight into the presence of additional peaks}\label{appendix:b}
Consider a linear, nearest neighbour, monotomic chain a with periodic boundary condition, the equations of motions are:
\begin{equation}
\begin{split}
m\ddot{x}_1&=k(x_2-x_1)+k(x_n-x_1)\\
m\ddot{x}_2&=k(x_3-x_2)+k(x_4-x_2)\\
...\\
m\ddot{x}_n&=k(x_{N-1}-x_N)+k(x_1-x_N)\\
\end{split}
\end{equation}
which is a set of coupled linear differential equations. Removing a single equation and the system becomes underdetermined (infinite number of solutions). The matrix version of this system is:
\begin{equation}
\bm{M}\bm{\ddot{X}}-\bm{K}\bm{X}=0
\end{equation}
With the usual harmonic assumption $\bm{X}=\bm{E}e^{i\omega t}$, the problem is recast:
\begin{equation}
[\omega^2\bm{M}^{-1}\bm{M}-\bm{M}^{-1}\bm{K}]\bm{E}=0
\end{equation}
Let
\begin{equation}
\bm{\psi}=
\begin{bmatrix}
   \bm{E}_1 & \bm{E}_2 & \dots &\bm{E}_n \\
 \end{bmatrix}
\end{equation}
We can write $\bm{\psi}$ as a transformation matrix
\begin{equation}
\begin{split}
\bm{X}&=\bm{\psi}\bm{Q}\\
\bm{Q}&=\bm{\psi}^{-1}\bm{X}
\end{split}
\end{equation}
In relation to the dividing the domain into subdomains, we want to determine $\bm{Q}$ using a reduced set of $\bm{X}$, given $\bm{\psi}$. Looking at a 4 MDOF case, the explicit matrix representation is:
\begin{equation}
\begin{bmatrix}
   Q_1\\
   Q_2\\
   Q_3\\
   Q_4\\
\end{bmatrix}=
\begin{bmatrix}
   E_{11} & E_{12} & E_{13} & E_{14} \\
   E_{21} & E_{22} & E_{23} & E_{24} \\
   E_{31} & E_{32} & E_{33} & E_{34} \\
   E_{41} & E_{42} & E_{43} & E_{44} \\
\end{bmatrix}^{-1}
\begin{bmatrix}
   x_1\\
   x_2\\
   x_3\\
   x_4\\
\end{bmatrix}
\end{equation}
Given a reduced set of $\bm{X}= [x_1,x_2]$, the equations are rearranged
\begin{equation}
\begin{split}
Q_1-E_{13}'x_3-E_{14}'x_4&= E_{11}'x_1 + E_{12}'x_2\\
Q_2-E_{23}'x_3-E_{24}'x_4&= E_{21}'x_1 + E_{12}'x_2\\
Q_3-E_{33}'x_3-E_{34}'x_4&= E_{31}'x_1 + E_{32}'x_2\\
Q_4-E_{43}'x_3-E_{44}'x_4&= E_{41}'x_1 + E_{42}'x_2.
\end{split}
\end{equation}
This results shows that the transformation from $\bm{X}$ to $\bm{Q}$ is relies upon missing information contained with $x_3$ and $x_4$. The original coupled equations of motions are:
\begin{align}
m\ddot{x}_1&=k(x_2-x_1)+k(x_4-x_1)\\
m\ddot{x}_2&=k(x_3-x_2)+k(x_1-x_2)\\
m\ddot{x}_3&=k(x_4-x_3)+k(x_2-x_3)\\
m\ddot{x}_4&=k(x_1-x_4)+k(x_3-x_4)
\end{align}
The eigenvalue problem takes the form:
\begin{equation}
\begin{bmatrix}
   m & 0 & 0 & 0 \\
   0 & m & 0 & 0 \\
   0 & 0 & m & 0 \\
   0 & 0 & 0 & m \\
\end{bmatrix}^{-1}
\begin{bmatrix}
   2k & -k & 0 & -k \\
   -k & 2k & -k & 0 \\
   0 & -k & 2k & -k \\
   -k & 0 & -k & 2k \\
\end{bmatrix}
\begin{bmatrix}
   E_{n,1} \\
   E_{n,2} \\
   E_{n,3} \\
   E_{n,4} \\
\end{bmatrix}
=\omega^2_n
\begin{bmatrix}
   E_{n,1} \\
   E_{n,2} \\
   E_{n,3} \\
   E_{n,4} \\
\end{bmatrix}
\end{equation}
Letting $m=1$ and $k=1$, the relation between coupled coordinates and normal mode coordinates is, $\pmb{\dot{X}}=\pmb{\psi}\pmb{\dot{Q}}$:
\begin{equation}
\begin{bmatrix}
   \dot{x}_1\\
   \dot{x}_2\\
   \dot{x}_3\\
   \dot{x}_4\\
\end{bmatrix}=
\begin{bmatrix}
  0.50000 & -0.50000  & 0.70711 & 0.00000 \\
  0.50000 &  0.50000 &  0.00000 & -0.70711 \\ 
  0.50000 & -0.50000 & -0.70711 & 0.00000  \\
  0.50000 &  0.50000 &  0.00000 & 0.70711  \\
\end{bmatrix}
\begin{bmatrix}
   \dot{Q}_1\\
   \dot{Q}_2\\
   \dot{Q}_3\\
   \dot{Q}_4\\
\end{bmatrix}
\end{equation}
If we take the 2 atom case:
\begin{equation}
\begin{bmatrix}
   m & 0  \\
   0 & m  \\
\end{bmatrix}^{-1}
\begin{bmatrix}
    k & -k  \\
   -k & k  \\
\end{bmatrix}
\begin{bmatrix}
   E_{n,1} \\
   E_{n,2} \\
\end{bmatrix}
=\omega^2_n
\begin{bmatrix}
   E_{n,1} \\
   E_{n,2} \\
\end{bmatrix}
\end{equation}
\begin{equation}
\begin{bmatrix}
   \dot{P}_1\\
   \dot{P}_2\\
\end{bmatrix}=
\begin{bmatrix}
  0.70711 &  -0.70711 \\
  0.70711 &  0.70711 \\
\end{bmatrix}^{-1}
\begin{bmatrix}
   \dot{x}_1\\
   \dot{x}_2\\
\end{bmatrix}
\end{equation}
Expressing $P$ in terms of $Q$:
\begin{align}
\dot{P}_1=\frac{\sqrt{2}\dot{Q}_1}{2}+\frac{\dot{Q}_3}{2}-\frac{\dot{Q}_4}{2}\\
\dot{P}_2=\frac{\sqrt{2}\dot{Q}_2}{2}-\frac{\dot{Q}_3}{2}-\frac{\dot{Q}_4}{2}
\end{align}
%P_1=\frac{\sqrt{2}Q_2}{2}+\frac{\sqrt{2}Q_4}{2}\\
%P_2=-\frac{\sqrt{2}Q_1}{2}-\frac{\sqrt{2}Q_3}{2}
Without taking $<\dot{Q}(\tau)\dot{Q}(0)>$, the Fourier transform of the set of $\dot{Q}$ using the output velocites from MD gives
\begin{figure}[!h]
\centering
%\includegraphics[trim = 40mm 0mm 10mm 0mm,scale=0.5,angle=0]{noxcorr.eps}
\includegraphics{noxcorr.eps}
\caption{Plots of power spectrums.}
\label{fig:awesome_image}
\end{figure}

