\chapter{Contributions and Future Research}

\section{Contributions}

Four methods (GK, DM, ALD, NMD) for predicting thermal conductivity were reviewed and applied to a bulk LJ argon system at 20 K. In Chapter~\ref{CHP:Interface}, an attempt was made to understand the effect on an interface upon individual phonon properties and confirmed the fact that NMD cannot be used on a portion of a domain. In Chapter~\ref{CHP:SL}, GK, NMD and ALD were applied to a series of LJ argon superlattices. It was found that bulk phonon properties were not representative of superlattice phonons for the selected range of superlattice period lengths. By including the secondary periodicity into the unit cell, and therefore the phonon dispersion, the resulting MFP spectrum and the trends in thermal conductivity as a function of superlattice period suggest that what previous literature refers to as a coherent effect may be equivalent to a simple dispersion effect. Furthermore, the application of Tamura theory and the use of perfect superlattice eigenvectors for mixed superlattices in the NMD procedure were found to yield comparable phonon lifetimes. In an effort to pursue open notebook science, all the code used and the data presented is freely available online\footnote{https://github.com/schuberm}.

\section{Future Research}

Future work will be to incorporate the HLD functionality from GULP directly into the LAMMPS package. This a crucial step if NMD is to be applied to other systems besides LJ and SW silicon. A topic that was only touched upon in this work is the understanding of size effects. Earlier studies relied upon macroscopic methods (i.e., GK and DM) to gain some insight into size effects. Due to the computational cost of a mode by mode analysis, NMD and ALD are conducted in such a way to use all available resources. As the algorithms, processing power and implementation improve, a rigorous study of the size effects upon each individual phonon mode will provide much needed insight. Another step will be to use DFT to develop empirical potentials for MD. This can be done by matching dispersion curves obtained from DFT to dispersion curves from MD.

