\chapter{Contributions and Future Research}

\section{Contributions}

Four methods (GK, DM, ALD, NMD) for predicting thermal conductivity were reviewed and applied to a bulk LJ argon system at 20 K. In Chapter~\ref{CHP:Interface}, an attempt was made to understand the effect of an interface upon individual phonon properties and confirmed the fact that NMD cannot be used on a portion of a domain. In Chapter~\ref{CHP:SL}, GK, NMD and ALD were applied to a series of LJ argon superlattices. It was found that bulk phonon properties were not representative of superlattice phonons for the selected range of superlattice period lengths. By including the secondary periodicity into the unit cell, and thereby into the phonon dispersion relation, the resulting MFP spectrum and the trends in thermal conductivity as a function of superlattice period suggest that what previous literature refers to as a coherent effect may be equivalent to a simple dispersion effect. Furthermore, the application of Tamura theory and the use of perfect superlattice eigenvectors for mixed superlattices in the NMD procedure were found to yield comparable phonon lifetimes. In an effort to pursue open notebook science, all the code used and the data presented is freely available online\footnote{https://github.com/schuberm}.

\section{Future Research}

The following list outlines some potential paths for future work.

\begin{itemize}
\item Incorporate the HLD functionality from GULP directly into the LAMMPS package. This a crucial step if NMD is to be applied to other systems besides LJ argon and Stillinger-Weber silicon.

\item A topic that was only touched upon in this work is the understanding of size effects. Earlier studies relied upon macroscopic methods (i.e., GK and DM) to gain some insight into size effects. Due to the computational cost of a mode by mode analysis, NMD and ALD are conducted in such a way to use all available resources. As the algorithms, processing power and implementation improve, a rigorous study of the size effects upon each individual phonon mode will provide much needed insight. 

\item Use DFT to develop empirical potentials for classical MD to improve the physical accuracy of thermal conductivity predictions. This can be done by matching dispersion curves obtained from DFT to the dispersion obtained from HLD. Furthermore, DFPT can be used to compare lifetimes with those from classical MD \cite{paulatto2013anharmonic}.

\item Incorporating disorder into the systems is necessary in order to make realistic thermal conductivity predictions. This will require working closely with experimentalists in order to characterize disorder \cite{millis2003towards}. 

\item Extend the approaches reviewed in this work to include the electron-phonon interaction \cite{PhysRevB.77.125209}. This is of importance in photovoltaic applications, where high rates of electron-phonon interaction limit the device efficiency.

\item Develop a hierarchical approach to resolve a multiscale perspective of thermal transport. This could be done by using data from DFT as input for MD, which can then be fed into a BTE solver \cite{pinedathesis}.
\end{itemize}

