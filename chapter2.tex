\chapter{Interface Study}

\section{Thermal Boundary Resistance}
The following section summarizes the seminal work of Swartz and Pohl on the challenges of measuring and modeling the thermal boundary resistance (TBR) \cite{RevModPhys.61.605}. TBR is defined as the ratio of the temperature discontinuity at the interface to the power per unit area flowing across that interface
\begin{equation}
TBR=\frac{T_l-T_r}{q_x}.
\end{equation}
The inverse of TBR is thermal boundary conductivity (TBC) which is defined as the ratio of heat flow per unit area to the temperature discontinuity at the interface. Kapitza was the first to report the measurement of this temperature discontinuity at the interface between helium and a solid.

To model the transmission of phonons across a solid-solid interface, Khalatnikov presented the acoustic mismatch model (AMM) and later modernized by Mazo and Onsager. The assumptions involved when using the AMM are a phonon is either transmitted across the interface or it is reflected, both sides of the interface are isotropic, the probability that a phonon is transmitted is independent of temperature and anharmonic interactions are ignored. The transmission probability is classically related to the fraction of energy that crosses the interface and acts as placeholder in the the energy balance of the interface. Within the AMM picture, phonons are interpreted as plane waves that move through a continuum; a perspective that allows one to view the transmission of phonons as anologous to the refraction of photons as they move from one medium to another. The major challenge in the AMM is to calculate the transmission probability for any incident angle and mode. Relying upon the acoustic analog to the Fresnel equations and assigning an impendance as function of the density of the material and the phonon velocity, an estimate of the transmission probability is obtained. The TBC is calculated by summing over all incoming angles and the occupancy of all phonons multiplied by the respective transmission probability.

Given the assumptions used in the AMM, it is not surprising that the results differ from experiments. The TBR between $^4$He and copper predicted by AMM was two orders of magnitude larger than experiment, a result which is not atypical for the standard AMM. In an attempt to resolve this discrepancy, Swartz proposed the Diffuse Mismatch Model (DMM) where, unlike the AMM in which phonons are specularly transmitted or reflected, all phonons are diffusely scattered, forward or backward, at the interface. Scattering is assumed to destroy any information about the phonon prior to scattering, where the possible modes available following scattering are determined by an energy balance and the phonon density of states. DMM tends to overestimate the amount of scattering at an interface. With such idealizations, AMM and DMM calculations of TBR are generally an order of magnitude different from experiment at temperatures above 30 K \cite{landrythesis}, but are useful to estimate an upper and lower limit (not necessarily in the respective order) to the TBR of a given interface system.

The incorporation of the inelastic scattering of phonons near and at the interface  into thermal transport models is required in order to improve predictions of TBR. Landry used anharmonic LD and the phonon BTE to determine the TBR across Si/Ge and Si/heavy-Si interfaces in non-equilibrium high-temperature conditions \cite{landrythesis}. The predicted TBRs were found to be 40-60\% less than the TBR predictions from MD with the direct method, where phonon scattering is inherently included This was one of the first results to conclusively demonstrate the importance of phonon scattering upon TBR. However, the effect upon phonon properties near an interface could not be elucidated from such an approach (Landry attributed the discrepancy in TBR predictions to deviations from the bulk phonon density of states, but did not examine anharmonic changes). The following section discusses the challenges of studying spatially-dependent phonon properties.

\section{NMD near an interface}

NMD was used in an attempt to observe the effect of an interface upon phonon properties, namely phonon lifetimes. Here, an interface is defined by enforcing a mass difference in a LJ argon system as seen in Fig. ~\ref{interface_domain}. In order to ensure the statistical significance of the results, an averaging scheme was needed. For each case, five independent MD simulations were conducted, each with a different initial velocity seed. In each MD simulation, 16 sets of atomic velocities were produced. Each set of velocities consisted of 2048 subsets of atomic velocities; the lag between subsets was 32 LJ time units (in other words, velocities were sampled every 32 LJ time units for a total of 2048 samples). The power of two formulation was chosen for the sake of the fast Fourier Transform. $\dot{Q}(\bm{\kappa},\nu,t)$ was calculated for each individual velocity subset (Eq.~\ref{}) and then used as input for the correlation function (Eq.~\ref{}) and SED (Equation \ref{}) of the set. Finally the SED is averaged over the 16 sets and 5 seeds. This procedure was performed on three cases: (A) a 4 unit cell by 4 unit cell by 4 unit cell (hereby referred to as $4\times4\times4$) domain of LJ argon in equilibrium at 20 K with periodic boundary conditions (B) a $32\times4\times4$ domain of LJ argon at 20 K with periodic boundary conditions and (C) $32\times4\times4$ domain of Lennard-Jones argon at 20 K, where one half ($16\times4\times4$) is set to the unit mass of argon and the other half is set to three times the unit mass of argon, with periodic boundary conditions. Cases B and C are divided into $4\times4\times4$ blocks for the post-processing steps to match the phonon modes present in Case A (see Fig. ~\ref{}), in an attempt to offer a mode by mode comparison. All MD simulations were performed at 20 K with periodic boundary conditions in all directions.
%%%
\begin{figure}%[ht!]
\begin{center}
\scalebox{0.75}{ \includegraphics{supcell_ai.eps}}
\renewcommand{\figure}{Fig.}
\caption{MD domain for interface study of $32\times4\times4$ FCC argon at 20K with 2048 atoms. Larger atomic radii represents the heavier mass.}
\label{fig:interface_domain}
\end{center}
\end{figure}
%%%
As the plane of the interface has a normal in the $x$ direction, it is reasonable to expect wavevectors containing some $x$ component to be affected. For simplicity, the modes at $\bm{\kappa}=[1,0,0]$ in the BZ are examined. 
%%%
\renewcommand{\topfraction}{1.0}
\begin{figure*}%[t]
\begin{center}
\scalebox{1}{ \includegraphics{sed_int.eps}}
\renewcommand{\figure}{Fig.}
\caption{Plots of example power spectrums.}
\label{fig:sed}
\end{center}
\end{figure*}
%%%
At first glance, it is clear that the SED of Cases B and C differ from Case A. The essence of the difference between Cases B and C and Case A lies not in the physics of the phonons, but in the model of their description. Ultimately, the idea of dividing the $32\times4\times4$ domains into $4\times4\times4$ blocks and performing NMD on these blocks to observe the change in the phonon lifetimes as a function of spatial position relative to the interface proved to be an ineffective approach.

The precise reason for these results is attributed to the mathematical nature of the problem. A sketch of the reason is offered here (a formal argument is presented in Appendix~\ref{appendix:b}). First, the definition of $Q(\bm{\kappa},\nu)$ requires the summation over all possible $\bm{\kappa},\nu$ of the entire $32\times4\times4$ domain. The division into blocks discounts the entire summation to only include available $\bm{\kappa},\nu$ of the $4\times4\times4$ block:
\begin{eqnarray}
\begin{split}
\dot{Q}(\bm{\kappa},\nu,t)-\frac{1}{\sqrt{N}}\sum_{28x4x4}\sqrt{m_j}exp(-i\bm{\kappa}\cdot\bm{r}(jl))\bm{e}^*(j,\bm{\kappa},\nu)\cdot\dot{\bm{u}}(jl,t)&=\\\frac{1}{\sqrt{N}}\sum_{4x4x4}\sqrt{m_j}exp(-i\bm{\kappa}\cdot\bm{r}(jl))\bm{e}^*(j,\bm{\kappa},\nu)\cdot\dot{\bm{u}}(jl,t).
\end{split}
\end{eqnarray}
Neglecting the modes that are uniqe to the complete $32\times4\times4$ domain, leads to an erroneous coordinate transformation from real space to $\pmb{\kappa}$-space because of the incompleteness of the solution basis. In other words, without using all the possible modes of the system to describe the motion of an individual atom, the corresponding normal mode cannot be inferred. The reverse is equally true.

Futhermore, the type of statistical ensemble of each individual block is difficult to ascertain. The entire domain is fixed to be NVE, but the energy of a block may not be fixed with time (in other words, the Hamiltonian of a block is not time-independent). At any given instant the energy of one block may be more or less than one of its neighbours, but the energy of all the blocks together remains constant.

This exercise revisits the struggle to handle any deviation from the perfectly periodic bulk crystal lattice. By relying on the eigenvectors and frequencies obtained from lattice dynamics, it is not obvious how to incorporate aperiodicities into this approach.


