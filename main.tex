%% ut-thesis.tex -- document template for graduate theses at UofT
%%
%% Copyright (c) 1998-2012 Francois Pitt <fpitt@cs.utoronto.ca>
%% last updated at 09:43 (EDT) on Fri  1 Jun 2012
%%
%% This work may be distributed and/or modified under the conditions of
%% the LaTeX Project Public License, either version 1.3c of this license
%% or (at your option) any later version.
%% The latest version of this license is in
%%     http://www.latex-project.org/lppl.txt
%% and version 1.3c or later is part of all distributions of LaTeX
%% version 2005/12/01 or later.
%%
%% This work has the LPPL maintenance status "maintained".
%%
%% The Current Maintainer of this work is
%% Francois Pitt <fpitt@cs.utoronto.ca>.
%%
%% This work consists of the files listed in the accompanying README.

%% SUMMARY OF FEATURES:
%%
%% All environments, commands, and options provided by the `ut-thesis'
%% class will be described below, at the point where they should appear
%% in the document.  See the file `ut-thesis.cls' for more details.
%%
%% To explicitly set the pagestyle of any blank page inserted with
%% \cleardoublepage, use one of \clearemptydoublepage,
%% \clearplaindoublepage, \clearthesisdoublepage, or
%% \clearstandarddoublepage (to use the style currently in effect).
%%
%% For single-spaced quotes or quotations, use the `longquote' and
%% `longquotation' environments.


%%%%%%%%%%%%         PREAMBLE         %%%%%%%%%%%%

%%  - Default settings format a final copy (single-sided, normal
%%    margins, one-and-a-half-spaced with single-spaced notes).
%%  - For a rough copy (double-sided, normal margins, double-spaced,
%%    with the word "DRAFT" printed at each corner of every page), use
%%    the `draft' option.
%%  - The default global line spacing can be changed with one of the
%%    options `singlespaced', `onehalfspaced', or `doublespaced'.
%%  - Footnotes and marginal notes are all single-spaced by default, but
%%    can be made to have the same spacing as the rest of the document
%%    by using the option `standardspacednotes'.
%%  - The size of the margins can be changed with one of the options:
%%     . `narrowmargins' (1 1/4" left, 3/4" others),
%%     . `normalmargins' (1 1/4" left, 1" others),
%%     . `widemargins' (1 1/4" all),
%%     . `extrawidemargins' (1 1/2" all).
%%  - The pagestyle of "cleared" pages (empty pages inserted in
%%    two-sided documents to put the next page on the right-hand side)
%%    can be set with one of the options `cleardoublepagestyleempty',
%%    `cleardoublepagestyleplain', or `cleardoublepagestylestandard'.
%%  - Any other standard option for the `report' document class can be
%%    used to override the default or draft settings (such as `10pt',
%%    `11pt', `12pt'), and standard LaTeX packages can be used to
%%    further customize the layout and/or formatting of the document.

%% *** Add any desired options. ***
\documentclass[12pt,doublespaced]{ut-thesis}

%% *** Add \usepackage declarations here. ***
%% The standard packages `geometry' and `setspace' are already loaded by
%% `ut-thesis' -- see their documentation for details of the features
%% they provide.  In particular, you may use the \geometry command here
%% to adjust the margins if none of the ut-thesis options are suitable
%% (see the `geometry' package for details).  You may also use the
%% \setstretch command to set the line spacing to a value other than
%% single, one-and-a-half, or double spaced (see the `setspace' package
%% for details).

\usepackage{graphicx}
\usepackage{epstopdf}
\usepackage{ifthen}
\usepackage{dcolumn}
\usepackage{bm}
\usepackage{multirow}
\usepackage{booktabs}
\usepackage{amsbsy}
\usepackage{amsmath}
\usepackage{amssymb}
\usepackage{subfigure}
\usepackage{booktabs}


%%%%%%%%%%%%%%%%%%%%%%%%%%%%%%%%%%%%%%%%%%%%%%%%%%%%%%%%%%%%%%%%%%%%%%%%
%%                                                                    %%
%%                   ***   I M P O R T A N T   ***                    %%
%%                                                                    %%
%%  Fill in the following fields with the required information:       %%
%%   - \degree{...}       name of the degree obtained                 %%
%%   - \department{...}   name of the graduate department             %%
%%   - \gradyear{...}     year of graduation                          %%
%%   - \author{...}       name of the author                          %%
%%   - \title{...}        title of the thesis                         %%
%%%%%%%%%%%%%%%%%%%%%%%%%%%%%%%%%%%%%%%%%%%%%%%%%%%%%%%%%%%%%%%%%%%%%%%%

%% *** Change this example to appropriate values. ***
\degree{Masters of Applied Science}
\department{Mechanical and Industrial Engineering}
\gradyear{2013}
\author{Samuel Huberman}
\title{Phonon Properties in Superlattices}

%% *** NOTE ***
%% Put here all other formatting commands that belong in the preamble.
%% In particular, you should put all of your \newcommand's,
%% \newenvironment's, \newtheorem's, etc. (in other words, all the
%% global definitions that you will need throughout your thesis) in a
%% separate file and use "\input{filename}" to input it here.

\newcolumntype{x}[1]{%
>{\centering\hspace{0pt}}p{#1}}%

%Definition of new commands
\newcommand{\f}[2]{\ensuremath{\frac{\displaystyle{#1}}{\displaystyle{#2}}}}
\newcommand{\lr}[1]{\langle{#1}\rangle}
\newcommand{\colv}[2] {\left(\begin{array}{c} #1 \\ #2 \end{array}\right)}
\renewcommand{\thefootnote}{\fnsymbol{footnote}}
\newcommand{\be} {\begin{eqnarray}}
\newcommand{\ee} {\end{eqnarray}}
%--------------------------------------------------------------------------
%EQ COMMANDS
%--------------------------------------------------------------------------
\newcommand{\two}{\mspace{-2.0mu}}
\newcommand{\four}{\mspace{-4.0mu}}
\newcommand{\plus}{\mspace{-4.5mu}+\mspace{-3.5mu}}
\newcommand{\minus}{\mspace{-4.5mu}-\mspace{-3.5mu}}
\newcommand{\pp}{'\mspace{-2.0mu}'}
\newcommand{\xlb}[4]{#1\ifthenelse{\equal{#2}{0}}{}{_{\alpha #2}}
\mspace{-2.0mu}\genfrac{(}{)}{0pt}{1}{\ifthenelse{\equal{#3}{0}}{0}{l #3}} 
{\ifthenelse{\equal{#4}{0}}{0}{b #4}}}

\newcommand{\xkv}[4]{#1\mspace{-5.0mu}\left(\mspace{-8.0mu}
\begin{smallmatrix}#2\four{}\four{}\mspace{-8.0mu}&\pmb{\kappa}#3\\&\nu 
#4\end{smallmatrix}\mspace{-5.0mu}\right)}

\newcommand{\evect}[6]{#1\mspace{-4.0mu}\left(\mspace{-8.0mu}
\begin{smallmatrix}#2\mspace{-8.0mu}&\pmb{\kappa} #3 &b #5\\&\nu #4 &
\alpha #6\end{smallmatrix}\mspace{-5.0mu}\right)}

\newcommand{\varmat}[8]{\mspace{-5.0mu}\left(\mspace{-8.0mu}
\begin{smallmatrix}\ifthenelse{\equal{#3}{0}}{\mspace{-8.0mu}&b_{#1}&b_{#2}
\\&\alpha_{#1}&\alpha_{#2}} {\ifthenelse{\equal{#7}{0}}{#1\mspace{-8.0mu}&
\pmb{\kappa}#2#3\mspace{-8.0mu}&\pmb{\kappa}#4#5\mspace{-8.0mu}&\pmb{\kappa}
#6\\&\nu#2&\nu#4&\nu#6} {#1\mspace{-8.0mu}&\pmb{\kappa}#2#3\mspace{-8.0mu}&
\pmb{\kappa}#4#5\mspace{-8.0mu}&\pmb{\kappa}#6#7\mspace{-8.0mu}&\pmb{\kappa}
#8\\&\nu#2&\nu#4&\nu#6&\nu#8}}\end{smallmatrix}\mspace{-5.0mu}\right)}

\newcommand{\EXP}[1]{\exp\mspace{-5.0mu}\left[#1\right]\mspace{-3.0mu}}

\newcommand{\tpp}[2]{\left(\mspace{-2.0mu}\xkv{\omega}{}{}{}#1\xkv{\omega}
{}{'}{'}#2\xkv{\omega}{}{\pp}{\pp}\mspace{-2.0mu}\right)}

%\newcommand{\SUM}[2]{\ifthenelse{\equal{#1}{0}}{\sum_{
%\alpha_{#2},b_{#2},l_{#2}}^{3,n,N}} {\ifthenelse{\equal{#1}{1}}{\sum_{
%\alpha_{#2},b_{#2}}^{3,n}}{\sum_{\pmb{\kappa}^{#2},\nu^{#2}}^{N,3n}}}}

\newcommand{\SUM[1]}{\sum_{\pmb{\kappa}^{#1},\nu^{#1}}^{N,3n}}

\newcommand{\SUMprime}[2]{\ifthenelse{\equal{#1}{0}}
{\sum_{\alpha_{#2},b_{#2},l_{#2}}^{3,n,N}} 
{\ifthenelse{\equal{#1}{1}}{\sum_{\alpha_{#2},b_{#2}}^{3,n}}
{\sum_{\pmb{\kappa}^{'}^{#2},\nu^{'}^{#2}}^{N,3n}}}}

\newcommand{\SUMalpha}[2]{\ifthenelse{\equal{#1}{0}}
{\sum_{\alpha_{#2}}^{3}} {\ifthenelse{\equal{#1}{1}}
{\sum_{\alpha_{#2},b_{#2}}^{3,n}}{\sum_{\pmb{\kappa}#2,\nu#2}^{N,3n}}}}

\newcommand{\SUMalphap}[2]{\ifthenelse{\equal{#1}{0}}
{\sum_{\alpha'_{#2}}^{3}} {\ifthenelse{\equal{#1}{1}}
{\sum_{\alpha'_{#2},b'_{#2}}^{3,n}}{\sum_{\pmb{\kappa}#2,\nu#2}^{N,3n}}}}

\newcommand{\SUMb}[2]{\ifthenelse{\equal{#1}{0}}{\sum_{b_{#2}}^{n}}
 {\ifthenelse{\equal{#1}{1}}{\sum_{\alpha_{#2},b_{#2}}^{3,n}}
{\sum_{\pmb{\kappa}#2,\nu#2}^{N,3n}}}}

\newcommand{\SUMbp}[2]{\ifthenelse{\equal{#1}{0}}{\sum_{b'_{#2}}^{n}}
 {\ifthenelse{\equal{#1}{1}}{\sum_{\alpha'_{#2},b'_{#2}}^{3,n}}
{\sum_{\pmb{\kappa}#2,\nu#2}^{N,3n}}}}

\newcommand{\SUMl}[2]{\ifthenelse{\equal{#1}{0}}{\sum_{l_{#2}}^{N}}
 {\ifthenelse{\equal{#1}{1}}{\sum_{\alpha_{#2},b_{#2}}^{3,n}}
{\sum_{\pmb{\kappa}#2,\nu#2}^{N,3n}}}}

\newcommand{\SUMlp}[2]{\ifthenelse{\equal{#1}{0}}{\sum_{l'_{#2}}^{N}}
 {\ifthenelse{\equal{#1}{1}}{\sum_{\alpha'_{#2},b'_{#2}}^{3,n}}
{\sum_{\pmb{\kappa}#2,\nu#2}^{N,3n}}}}

\newcommand{\abcdt}[5]{\mspace{-4.0mu}\left(\mspace{-8.0mu}
\begin{smallmatrix}&\ifthenelse{\equal{#1}{}}{a}{#1}&\ifthenelse
{\equal{#3}{}}{c}{#3}\\&\ifthenelse{\equal{#2}{}}{b}{#2}&\ifthenelse
{\equal{#4}{}}{d}{#4}\end{smallmatrix}\mspace{-2.0mu};\ifthenelse
{\equal{#5}{}}{t}{#5}\right)}

\newcommand{\abcd}[4]{\mspace{-4.0mu}\left(\mspace{-8.0mu}
\begin{smallmatrix}&\ifthenelse{\equal{#1}{}}{a}{#1}&\ifthenelse
{\equal{#3}{}}{c}{#3}\\&\ifthenelse{\equal{#2}{}}{b}{#2}&\ifthenelse
{\equal{#4}{}}{d}{#4}\end{smallmatrix}\mspace{-3.0mu}\right)}

\newcommand{\abt}[3]{\mspace{-4.0mu}\left(\mspace{-8.0mu}\begin
{smallmatrix}&\ifthenelse{\equal{#1}{}}{a}{#1} \\&\ifthenelse{
\equal{#2}{}}{b}{#2}\end{smallmatrix}\mspace{-2.0mu};
\ifthenelse{\equal{#3}{}}{t}{#3}\right)}

\newcommand{\ab}[2]{\mspace{-4.0mu}\left(\mspace{-8.0mu}
\begin{smallmatrix}&\ifthenelse{\equal{#1}{}}{a}{#1} \\&\ifthenelse
{\equal{#2}{}}{b}{#2}\end{smallmatrix}\mspace{-3.0mu}\right)}

\newcommand{\kvbat}{\mspace{-4.0mu}\left(\mspace{-8.0mu}
\begin{smallmatrix} &\pmb{\kappa} &b \\ &\nu &\alpha\end{smallmatrix}
\mspace{-2.0mu};t\right)}

\newcommand{\kvbatp}{\mspace{-4.0mu}\left(\mspace{-8.0mu}
\begin{smallmatrix} &\pmb{\kappa} &b' \\ &\nu &\alpha'\end{smallmatrix}
\mspace{-2.0mu};t\right)}

\newcommand{\kvbaw}{\mspace{-4.0mu}\left(\mspace{-8.0mu}
\begin{smallmatrix} &\pmb{\kappa} &b \\ &\nu &\alpha\end{smallmatrix}
\mspace{-2.0mu};\omega\right)}

\newcommand{\kvbawp}{\mspace{-4.0mu}\left(\mspace{-8.0mu}
\begin{smallmatrix} &\pmb{\kappa} &b' \\ &\nu &\alpha'\end{smallmatrix}
\mspace{-2.0mu};\omega\right)}

\newcommand{\kvba}{\mspace{-4.0mu}\left(\mspace{-8.0mu}
\begin{smallmatrix} &\pmb{\kappa} &b \\ &\nu &\alpha\end{smallmatrix}
\mspace{-3.0mu}\right)}

\newcommand{\kvbain}[1]{\mspace{-4.0mu}\left(\mspace{-8.0mu}
\begin{smallmatrix} &\pmb{\kappa}^{#1}  &b^{#1} \\ &\nu^{#1}  &\alpha^{#1} \end{smallmatrix} \mspace{-3.0mu}\right)}

\newcommand{\kvbap}{\mspace{-4.0mu}\left(\mspace{-8.0mu}
\begin{smallmatrix} &\pmb{\kappa}' &b \\ &\nu' &\alpha\end{smallmatrix}
\mspace{-3.0mu}\right)}

\newcommand{\kpvba}{\mspace{-4.0mu}\left(\mspace{-8.0mu}
\begin{smallmatrix} &\pmb{\kappa}^{'} &b \\ &\nu &\alpha\end{smallmatrix}
\mspace{-3.0mu}\right)}

\newcommand{\kva}{\mspace{-4.0mu}\left(\mspace{-8.0mu}
\begin{smallmatrix} &\pmb{\kappa} \\ &\nu &\alpha\end{smallmatrix}
\mspace{-3.0mu}\right)}

\newcommand{\kvap}{\mspace{-4.0mu}\left(\mspace{-8.0mu}
\begin{smallmatrix} &\pmb{\kappa} \\ &\nu &\alpha'\end{smallmatrix}
\mspace{-3.0mu}\right)}

\newcommand{\kvb}{\mspace{-4.0mu}\left(\mspace{-8.0mu}
\begin{smallmatrix} &\pmb{\kappa} \\ &\nu \end{smallmatrix}
\mspace{-2.0mu},b\right)}

\newcommand{\kvbp}{\mspace{-4.0mu}\left(\mspace{-8.0mu}
\begin{smallmatrix} &\pmb{\kappa} &b' \\ &\nu \end{smallmatrix}
\mspace{-3.0mu}\right)}

\newcommand{\kvt}{\mspace{-4.0mu}\left(\mspace{-8.0mu}
\begin{smallmatrix}&\pmb{\kappa} \\&\nu\end{smallmatrix}
\mspace{-2.0mu},t\right)}

\newcommand{\kvzero}{\mspace{-4.0mu}\left(\mspace{-8.0mu}
\begin{smallmatrix}&\pmb{\kappa} \\&\nu\end{smallmatrix}
\mspace{-2.0mu};0\right)}

\newcommand{\kpvt}{\mspace{-4.0mu}\left(\mspace{-8.0mu}
\begin{smallmatrix}&\pmb{\kappa}^{'} \\&\nu\end{smallmatrix}
\mspace{-2.0mu};t\right)}

\newcommand{\kvw}{\mspace{-4.0mu}\left(\mspace{-8.0mu}
\begin{smallmatrix}&\pmb{\kappa} \\&\nu\end{smallmatrix}
\mspace{-2.0mu},\omega\right)}

\newcommand{\kv}{\mspace{-4.0mu}\left(\mspace{-8.0mu}
\begin{smallmatrix}&\pmb{\kappa} \\&\nu\end{smallmatrix}
\mspace{-3.0mu}\right)}

\newcommand{\kvp}{\mspace{-4.0mu}\left(\mspace{-8.0mu}
\begin{smallmatrix}&\pmb{\kappa'} \\&\nu'\end{smallmatrix}
\mspace{-3.0mu}\right)}

\newcommand{\kvpp}{\mspace{-4.0mu}\left(\mspace{-8.0mu}
\begin{smallmatrix}&\pmb{\kappa''} \\&\nu''\end{smallmatrix}
\mspace{-3.0mu}\right)}

\newcommand{\kw}{\mspace{-4.0mu}\left(\mspace{-8.0mu}
\begin{smallmatrix}&\pmb{\kappa} \\&\omega\end{smallmatrix}
\mspace{-3.0mu}\right)}

\newcommand{\kpvp}{\mspace{-4.0mu}\left(\mspace{-8.0mu}
\begin{smallmatrix}&\pmb{\kappa'} \\&\nu'\end{smallmatrix}
\mspace{-3.0mu}\right)}

\newcommand{\kvkvpkvpp}{\mspace{-4.0mu}\left(\mspace{-8.0mu}
\begin{smallmatrix}&\pmb{\kappa}&\pmb{\kappa'}&\pmb{\kappa''} \\&\nu&\nu'&\nu''\end{smallmatrix}
\mspace{-3.0mu}\right)}

\newcommand{\kvOv}{\mspace{-4.0mu}\left(\mspace{-8.0mu}
\begin{smallmatrix}&\pmb{\kappa}&\pmb{-\kappa}&0 \\&\nu&\nu&\nu''\end{smallmatrix}
\mspace{-3.0mu}\right)}

\newcommand{\kvpOv}{\mspace{-4.0mu}\left(\mspace{-8.0mu}
\begin{smallmatrix}&\pmb{\kappa'}&\pmb{-\kappa'}&0 \\&\nu'&\nu'&\nu''\end{smallmatrix}
\mspace{-3.0mu}\right)}

\newcommand{\Ovpp}{\mspace{-4.0mu}\left(\mspace{-8.0mu}
\begin{smallmatrix}&\pmb{0} \\&\nu''\end{smallmatrix}\mspace{-3.0mu}\right)}

\newcommand{\lbt}{\mspace{-4.0mu}\left(\mspace{-8.0mu}
\begin{smallmatrix}&l \\&b\end{smallmatrix}\mspace{-2.0mu},t\right)}

\newcommand{\lbtp}{\mspace{-4.0mu}\left(\mspace{-8.0mu}
\begin{smallmatrix}&l' \\&b'\end{smallmatrix}\mspace{-2.0mu},t\right)}

\newcommand{\lt}{\mspace{-4.0mu}\left(\mspace{-8.0mu}
\begin{smallmatrix}&l\end{smallmatrix}\mspace{-2.0mu},t\right)}

\newcommand{\ltp}{\mspace{-4.0mu}\left(\mspace{-8.0mu}
\begin{smallmatrix}&l'\end{smallmatrix}\mspace{-2.0mu},t\right)}

\newcommand{\lb}{\mspace{-4.0mu}\left(\mspace{-8.0mu}
\begin{smallmatrix}&l \\&b\end{smallmatrix}\mspace{-3.0mu}\right)}

\newcommand{\Ob}{\mspace{-4.0mu}\left(\mspace{-8.0mu}
\begin{smallmatrix}&0 \\&b\end{smallmatrix}\mspace{-3.0mu}\right)}

\newcommand{\lO}{\mspace{-4.0mu}\left(\mspace{-8.0mu}
\begin{smallmatrix}&l \\&0\end{smallmatrix}\mspace{-3.0mu}\right)}

\newcommand{\lOin}[1]{\mspace{-4.0mu}\left(\mspace{-8.0mu}
\begin{smallmatrix}&l^{#1} \\&0\end{smallmatrix}\mspace{-3.0mu}\right)}

\newcommand{\lbp}{\mspace{-4.0mu}\left(\mspace{-8.0mu}
\begin{smallmatrix}&l' \\&b'\end{smallmatrix}\mspace{-3.0mu}\right)}

\newcommand{\lbpp}{\mspace{-4.0mu}\left(\mspace{-8.0mu}
\begin{smallmatrix}&l'' \\&b''\end{smallmatrix}\mspace{-3.0mu}\right)}

\newcommand{\lblb}{\mspace{-4.0mu}\left(\mspace{-8.0mu}
\begin{smallmatrix}&l_1&l_2\\&b_1&b_2\end{smallmatrix}
\mspace{-3.0mu}\right)}

\newcommand{\lblblb}{\mspace{-4.0mu}\left(\mspace{-8.0mu}
\begin{smallmatrix}&l_1&l_2&l_3\\&b_1&b_2&b_3\end{smallmatrix}
\mspace{-3.0mu}\right)}



%% *** Adjust the following settings as desired. ***

%% List only down to subsections in the table of contents;
%% 0=chapter, 1=section, 2=subsection, 3=subsubsection, etc.
\setcounter{tocdepth}{2}

%% Make each page fill up the entire page.
\flushbottom


%%%%%%%%%%%%      MAIN  DOCUMENT      %%%%%%%%%%%%

\begin{document}

%% This sets the page style and numbering for preliminary sections.
\begin{preliminary}

%% This generates the title page from the information given above.
\maketitle

%% There should be NOTHING between the title page and abstract.
%% However, if your document is two-sided and you want the abstract
%% _not_ to appear on the back of the title page, then uncomment the
%% following line.
%\cleardoublepage

%% This generates the abstract page, with the line spacing adjusted
%% according to SGS guidelines.
\begin{abstract}
%% *** Put your Abstract here. ***
%% (At most 150 words for M.Sc. or 350 words for Ph.D.)
\end{abstract}

%% Anything placed between the abstract and table of contents will
%% appear on a separate page since the abstract ends with \newpage and
%% the table of contents starts with \clearpage.  Use \cleardoublepage
%% for anything that you want to appear on a right-hand page.

%% This generates a "dedication" section, if needed
%% (uncomment to have it appear in the document).
%\begin{dedication}
%% *** Put your Dedication here. ***
%\end{dedication}

%% The `dedication' and `acknowledgements' sections do not create new
%% pages so if you want the two sections to appear on separate pages,
%% you should put an explicit \newpage between them.

%% This generates an "acknowledgements" section, if needed
%% (uncomment to have it appear in the document).
%\begin{acknowledgements}
%% *** Put your Acknowledgements here. ***
%\end{acknowledgements}

%% This generates the Table of Contents (on a separate page).
\tableofcontents

%% This generates the List of Tables (on a separate page), if needed
%% (uncomment to have it appear in the document).
%\listoftables

%% This generates the List of Figures (on a separate page), if needed
%% (uncomment to have it appear in the document).
%\listoffigures

%% You can add commands here to generate any other material that belongs
%% in the head matter (for example, List of Plates, Index of Symbols, or
%% List of Appendices).

%% End of the preliminary sections: reset page style and numbering.
\end{preliminary}


%%%%%%%%%%%%%%%%%%%%%%%%%%%%%%%%%%%%%%%%%%%%%%%%%%%%%%%%%%%%%%%%%%%%%%%%
%%  Put your Chapters here; the easiest way to do this is to keep     %%
%%  each chapter in a separate file and `\include' all the files.     %%
%%  Each chapter file should start with "\chapter{ChapterName}".      %%
%%  Note that using `\include' instead of `\input' will make each     %%
%%  chapter start on a new page, and allow you to format only parts   %%
%%  of your thesis at a time by using `\includeonly'.                 %%
%%%%%%%%%%%%%%%%%%%%%%%%%%%%%%%%%%%%%%%%%%%%%%%%%%%%%%%%%%%%%%%%%%%%%%%%

%% *** Include chapter files here. ***

\chapter{Review of Thermal Conductivity Prediction Methods}

Four methods to predict thermal conductivity are presented and a comparison of bulk LJ is offered. The methods are divided into two categories: macroscopic oriented (Green-Kubo and Direct Method) and mode oriented methods (Anharmonic Lattice Dynamics and Normal Mode Decomposition).

\section{Macroscopic Methods}
\subsection{Green-Kubo}
\subsubsection{Fluctuation-Dissipation Theorem}

The Fluctuation-Dissipation Theorem (FDT), first stated by Harry Nyquist (in relation to the Nyquist noise), was later reformulated by Ryogo Kubo to relate transport coefficients to time correlation functions \cite{JPSJ.12.570} under an external force (it is worthwhile to distinguish between external and internal forces, electrical conductivity under and applied electric field versus the viscosity of a fluid). \cite{} What follows is an outline of the logic of Kubo's generalization of the FDT using Helfand's approach as reviewed by McQuarrie\cite{mcquarrie}. The statistical version of the diffusion equation
%
\begin{equation}
\begin{split}
	\frac{\partial G(\bm{r},t)}{\partial t}&= D \nabla ^2G(\bm{r},t)\\
G(\bm{r},0)&=\delta(\pm{r})
\end{split}
\end{equation}
%
Here, $G(\bm{r},t)$ is the fraction of particles in phase coordinates about $d\bm{r}$ at $\bm{r}$ at time $t$ given that they were located at $\bm{r}(0)$ at $t=0$
%
\begin{equation}
G(\bm{r},t)= \frac{1}{N}\left<\sum_{j=1}^N\delta(\bm{r}-[\bm{r}_j(t)-\bm{r}_j(0)])\right>.
\end{equation}
%
Einstein was the first to show that the solution to this diffusion equation is
%
\begin{equation}
\left<[\bm{r}(t)-\bm{r}(0)]^2\right>=6Dt
\end{equation}
%
%At this point, it is insightful to take the Fourier transform of $G(\bm{r},t)$ as it provides the form of the time correlation function, $<A^*(t)A(0)>$
%
%\begin{equation}
%F(\bm{\kappa},t)=\int_{-\infty}^{\infty}G(\bm{r},t)e^{-i\bm{\kappa}\cdot\bm{r}}d\bm{r}=\frac{1}{N}\left<\sum_{j=1}^Ne^{i\bm{\kappa}\cdot\bm{r}_j(t)}e^{-i\bm{\kappa}\cdot\bm{r}_j(0)}\right>.
%\end{equation}
%\begin{equation}
%\begin{split}
%	F_s(\pm{k},t)&=\int e^{i \pm{k} \cdot \pm{r}}G(\bm{r},t) d\pm{r}\\
%		     &=e^{-k^2Dt}
%\end{split}
%\end{equation}
%
where $<A>$ refers to the ensemble average of observable A
\begin{equation}
\int \int A(p,q) f(p,q) dpdq=\left<A\right>
\end{equation}
where $<A(t)A(0)>$ refers to the equilibrium time correlation of observable A
\begin{equation}
\left<A(\delta t)A(0)\right>=\frac{1}{N}\sum_{i=1}^NA(t_i)A(t_i+\delta t) 
\end{equation}
From the ergodic hypothesis
\begin{equation}
\left<A( t)A(0)\right>=\int \int A(t,p,q)A(0,p,q) f(p,q) dpdq 
\end{equation}
Only classical systems are considered here, but these analogies are easily extended to quantum mechanics (the ensemble average is the trace of the operator).To show how these correlation functions manifest themselves, recall the basic kinematic relations in the most general forms
%
\begin{equation}
\bm{r}(t)-\bm{r}(0)=\int_0^t \bm{v}(t')dt'
\end{equation}
\begin{equation}
[\bm{r}(t)-\bm{r}(0)]^2=\int_0^t \int_0^t \bm{v}(t')\cdot\bm{v}(t'')dt'dt''.
\end{equation}
%
The usefulness of Eq. \ref {} becomes clear upon taking the ensemble average thus giving the average behaviour of all possible functions of the phase coordinates $\bm{r},\bm{v}$
%
\begin{equation}
<[\bm{r}(t)-\bm{r}(0)]^2>=\int_0^t \int_0^t <\bm{v}(t')\cdot\bm{v}(t'')>dt'dt''.
\end{equation}
%
Since the ensemble is a stationary process (e.i.: independent of the definition of the origin of time) and the classical equations of motion are time-symmetric
%
\begin{equation}
<\bm{v}(t')\cdot\bm{v}(t'')>=<\bm{v}(t'-t'')\cdot\bm{v}(0)>=<\bm{v}(t''-t')\cdot\bm{v}(0)>.
\end{equation}
%
From the initial solution to the diffusion equation, substituting $\tau=t''-t'$ and performing the first integration
%
\begin{equation}
6Dt=2t\int_0^t\left(1-\frac{\tau}{t}\right)<\bm{v}(\tau)\cdot\bm{v}(0)>d\tau.
\end{equation}
%
Assuming that the time correlation function decays to zero long before $t$, the final form can be taken
%
\begin{equation}
D=\frac{1}{3}\int_0^{\infty}<\bm{v}(\tau)\cdot\bm{v}(0)>d\tau.
\end{equation}
%
and thus arriving at an expression for the diffusion coefficient in terms of the equilibrium time correlation function for velocity (time dependence of an ensemble average). Kubo generalized this result by expressing the linear response of a system from small perturbations in terms of its fluctuations about equilibrium. It can be applied to predict thermal conductivity (as will be seen)\cite{} and electrical conductivity, \cite{} viscosity \cite{}

\subsubsection {Molecular Dynamics and Green-Kubo}

The GK relation for thermal conductivity falls out of the fluctuation-dissipation theorem and the assumptions made therein, namely that the perturbations to the system's Hamiltonian are small and that the stochastic processes are Markoffian \cite{green:398}. Thus the thermal conductivity can be related to the fluctuations of the heat current vector, $\bm{S}$, over long periods
%
\begin{equation}\label{EQ:intHCACF}
k=\frac{1}{k_B V T^2}\int_0^{\infty}\frac{<\bm{S}(t)\cdot\bm{S}(0)>}{3}dt.
\end{equation}
%
The heat current vector is given by 
%
\begin{equation}\label{EQ:HCvec}
\bm{S}=\frac{d}{dt}\sum_i\bm{r}_iE_i
\end{equation}
%
where $E_i$ is the energy of particle $i$ at position $\bm{r}_i$. $<\bm{S}(t)\cdot\bm{S}(0)>$ is referred to as the heat current autocorrelation function (HCACF). For pairwise potentials, the heat current vector is
%
\begin{equation}\label{EQ:HCvec_pair}
\bm{S}=\sum_ie_i\bm{v}_i+\frac{1}{2}\sum_{i,j}(\bm{F}_{ij}\cdot\bm{v}_{i})\bm{r}_{ij}.
\end{equation}
%
Eq.~\ref{EQ:HCvec} and~\ref{EQ:HCvec_pair} can be easily added to a MD code since the quantities with which $\bm{S}$ is calculated, typically, are tracked for every time step. McGaughey examined the time dependence of the HCACF and noted a two stage behaviour for crystals: a rapid initial decay corresponding to the damping of the fluctuations and a slow secondary oscillatory decay, which is believed to be associated with the periodic boundary conditions of the simulation. These oscillations decreased as the simulation size was increased \cite{mcgaugheythesis}.
%
\begin{figure}
\begin{center}
\scalebox{1}{ \includegraphics{GK_bulk.eps}}
\renewcommand{\figure}{Fig.}
\caption{Integral of HCACF (right axis) and relative error function (left axis).}
\label{FIG:GK_bulk}
\end{center}
\end{figure}
%
For cases where the HCACF converges well, the thermal conductivity can be found by numerically integrating over a suitable range. Li et al. \cite{Li1998139} use two methods to determine objectively determine the definition of suitable range. One method is simply to evaluate the integral to the point where the HCACF first becomes negative, known as the first dip method. For cases where the HCACF remains positive, an exponential fit is used to determine the contribution of the tail.

In the case of amorphous materials, the HCACF does not converge prior to becoming negative, thus the first dip or exponential fit methods cannot be used and \textit{a priori} knowledge of the functional form of the HCACF is required in order to complete the integration and predict thermal conductivity.

\subsection{The Direct Method}

Unlike the GK method, the direct method uses a non-equilibrium steady-state approach to determine thermal conductivity. A one-dimension heat flux is generated through the simulation, typically by keeping the boundaries of the simulation at fixed, but different, temperatures, such that the boundaries behave like a hot and cold thermodynamic bath. From Fourier's law, the thermal conductivity can be predicted once the heat flux has converged
%
\begin{equation}\label{EQ:DM_k}
k=-\frac{q_x}{dT/dx}.
\end{equation}
%
Equivalently, a heat flux can be imposed and the corresponding temperature difference calculated. Generally, both set-ups are used, but the time to convergence of the heat flux vector is orders of magnitude greater than that of the temperature difference.

The applicability of direct method is questionable in situations where the temperature profile is not linear. Such is the case for nanoscale systems with a temperature difference on the order of 10K \cite{mcgaugheythesis}.
%
\begin{figure}
\begin{center}
\scalebox{1}{ \includegraphics{DM_bulk.eps}}
\renewcommand{\figure}{Fig.}
\caption{Temperature profile from the Direct Method simulation of bulk LJ argon at 20 K.}
\label{FIG:GK_bulk}
\end{center}
\end{figure}
%
%
\section{Mode by Mode Methods}
\subsection{SMRT-BTE}
%
In order to overcome the limitations of the DM and GK method, scientists have turned to the Boltzmann Transport Equation (BTE).\cite{srivastava1990physics} 
%
\begin{equation}\label{EQ:BTE}
-c(\pmb{\kappa})\cdot \nabla T\frac{\partial f\kv}{\partial T}=\frac{\partial f\kv}{\partial t}
\end{equation}
%
Although analytically intractable, there exists a multitude of computational solutions for the BTE. Under the relaxation time approximation
%
\begin{equation}\label{EQ:BTE_lin}
-c(\pmb{\kappa})\cdot \nabla T\frac{\partial f\kv}{\partial T}=\frac{f\kv-f_{eq}\kv}{\tau \kv}
\end{equation}
%
The thermal conductivity, under the RTA, can thus be expressed in terms of contributions over all possible phonon modes \cite{srivastava1990physics}
%
\begin{equation}\label{EQ:k_RTA}
	k_{z}= \sum_\nu\sum_\kappa c_{ph}\kv v^2_{g,z}\kv\tau_{p-p}\kv.
\end{equation}
%
Here $c_{ph}\kv$ is the volumetric specific heat from the classical thermodynamic definition
%
\begin{equation}\label{EQ:Cph}
c_{ph}\kv=\frac{\partial E}{V\partial T}=\frac{\hbar\omega\kv}{V}\frac{\partial f^{BE}\kv}{\partial T}	
\end{equation}
%
and
\begin{equation}\label{EQ:Vg}
v_{g,z}\kv=\frac{\partial \omega\kv}{\partial \bm{\kappa}}
\end{equation}
is the group velocity in the $z$ direction which can readily obtained from harmonic lattice dynamics. 

\subsection{Lattice Dynamics}
\subsubsection{Harmonic Lattice Dynamics}
%\begin{figure}[ht]
%\centering
%\includegraphics[scale=0.5]{diatomic.png}
%\caption{Diagram of a linear diatomic chain of atoms}
%\end{figure}
To illustrate the fundamental concepts applied in lattice dynamics calculations, the following 1D case is considered. Recalling the equations of motion of a linear diatomic chain considering only nearest neighbour interaction ($K_1$ and $K_2$ are the respective spring constants in accordance with Hooke's Law) \cite{dove_introduction_1993-3}:
%
\begin{align}
%\begin{split}
	M\frac{\partial ^2 u_n}{\partial t^2}&=-K_1(U_n-u_n)-K_2(U_n-u_{n-1})\\
	m\frac{\partial ^2 u_n}{\partial t^2}&=-K_1(u_n-U_n)-K_2(u_n-U_{n+1})
%\end{split}
\end{align}
%
Recognizing that the solutions to the equations will have the periodic form:
%
\begin{align}
	U_n&=\sum_\kappa \tilde{U_\kappa}e^{i(\kappa na-\omega t)}\\
	u_n&=\sum_\kappa \tilde{u_\kappa}e^{i(\kappa na-\omega t)}
\end{align}
%
Upon the application of the proper boundary conditions (the Born-von Karman periodic boundary, where the last atom in the chain is equivalent to the first in the chain), a discrete set of allowed values of $\kappa$ emerges (this condition is not affected by the fact that the masses of the chain alternate):
%
\begin{align}
	\pmb{\kappa}=\frac{2\pi m}{Na}
\end{align}
%
Each $\kappa$ in this set corresponds to a different phonon mode. The two coupled ordinary differential equations can be represented in terms of an eigenvalue problem:
%
\begin{equation}
\begin{bmatrix}
  -M\omega_\kappa^2 & 0\\
  0 & -m\omega_\kappa^2\\ 
 \end{bmatrix}
\begin{bmatrix}
\tilde{U_\kappa} \\ 
\tilde{u_\kappa}
\end{bmatrix}
=
\begin{bmatrix}
  -(K_1+K_2) & K_1+K_2e^{-i\kappa a}\\
  -(K_1+K_2) & K_1+K_2e^{-i\kappa a}\\ 
 \end{bmatrix}
\begin{bmatrix}
\tilde{U_\kappa} \\ \tilde{U_\kappa}
\end{bmatrix}
\end{equation}
%
The eigenvalues are the allowed frequencies and the eigenvectors are the allowed amplitudes for a given value of $\pmb{\kappa}$. The range of wavevectors, which operate in reciprocal space,  $\frac{-\pi}{a}\leq \pmb{\kappa}\leq\frac{\pi}{a}$ gives the first Brillouin zone. Because of the periodic nature of the lattice and hence $\pmb{\kappa}$, values outside this region may be folded back over so as to be included in the first Brillouin zone.
Applying this approach to a genuine lattice structure requires information about the interatomic potentials. For silicon, the empirical Stilinger-Weber potential is typically used in molecular dynamic simulations. For rare gas solids like argon, the Lennard-Jones potential describes the interatomic energy
\begin{align}
	\phi(r)=-4\epsilon[(\frac{\sigma}{r})^6-(\frac{\sigma}{r})^{12}]	
\end{align}
The spring constants in the equations of motions of the atoms of the lattice can then be calculated by expanding the energy of the lattice in Taylor series:
%
\begin{align}
	E&=N\phi(a)\\
	E&=N\phi+\sum_{s\geq1}\frac{1}{s!}\frac{\partial^s\phi}{\partial u^s}\sum_n(u_n-u_{n+1})^s
\end{align}
%
The harmonic approximation is the truncation of this expansion, neglecting orders greater than two. From this expansion, we find that the force constants must be:
%
\begin{align}
	K_1&=\frac{\partial^2 E}{\partial u_n\partial U_{n}}\\
	K_2&=\frac{\partial^2 E}{\partial u_n\partial U_{n+1}}
\end{align}
%
Under this approximation, the force acting on atom $a$ as a result of displacement of atom $b$ in the unit cell is the second derivative of the potential with respect to the displacement of both atoms. This definition extends to three dimensional structures by including the direction of the displacement and the force.
%
\begin{equation}
\Phi_{ab}=
\begin{bmatrix}
  \frac{\partial^2 \phi}{\partial u^a_i\partial u^b_i} & \frac{\partial^2 \phi}{\partial u^a_i\partial u^b_j} &\frac{\partial^2 \phi}{\partial u^a_i\partial u^b_k}\\
  \frac{\partial^2 \phi}{\partial u^a_j\partial u^b_i} & \frac{\partial^2 \phi}{\partial u^a_j\partial u^b_j} &\frac{\partial^2 \phi}{\partial u^a_j\partial u^b_k}\\
\frac{\partial^2 \phi}{\partial u^a_k\partial u^b_i} & \frac{\partial^2 \phi}{\partial u^a_k\partial u^b_j} &\frac{\partial^2 \phi}{\partial u^a_k\partial u^b_k}
 \end{bmatrix}
\end{equation}
Provided with the knowledge of these harmonic force constants, the general form of the eigenvalue problem is then constructed \cite{dove_introduction_1993-3}
\begin{equation}
[D(\pmb{\kappa})-I\omega^2\kv]\pmb{e}\kv = 0
\end{equation}
%
The dynamical matrix, $D(\pmb{\kappa})$ contains the chunks of three by three force constants from $\Phi_{ab}$ as well as the time independent portion of the general solution form and as such depends upon the wavevector
\begin{equation}
D_{3(b-1)+\alpha,3(b'-1)+\alpha'}(\pmb{\kappa})=\frac{1}{\sqrt{m_bm_{b'}}}\sum_{l'}^N\frac{\partial^2}{\partial r_\alpha \Ob \partial r_{\alpha'} \lbp} \EXP{i\pmb{\kappa}\cdot [\pmb{r}\lbp-\pmb{r}\Ob]}.
\end{equation}
Solving this matrix equation over a grid of wavevectors in the first Brillouin zone, provides a set of data where frequency is a function of wavector known formerly as dispersion relations (unlike the diatomic case, there more than two possible branches, $\nu$, as a result of the greater number of degrees of freedom of the atoms in the unit cell). 

\subsubsection{Anharmonic Lattice Dynamics}

The natural extension is to apply perturbation theory \cite{turneythesis}.
%
\begin{equation}\label{EQ:Gamma_anh}
\begin{split}
\Gamma \kv &= \\
&\frac{\pi\hbar}{16N}\SUM[']\SUM['']|\Phi\kvkvpkvpp|^2[[f_0\kvp+f_0\kvpp+1][\delta\left(\omega\kv-\omega\kvp-\omega\kvpp\right)-\delta\left(\omega\kv+\omega\kvp+\omega\kvpp\right)]\\
&+[f_0\kvp-f_0\kvpp+1][\delta\left(\omega\kv+\omega\kvp+\omega\kvpp\right)-\delta\left(\omega\kv-\omega\kvp-\omega\kvpp\right)]]\\
&+\frac{\pi\hbar}{8N}\SUM[']\sum_{\nu''}^{3n}\Phi\kvOv\Phi(\kvpOv[2f_0\kvp+1]\delta(\omega\Ovpp)
\end{split}
\end{equation}
%
where
%
\begin{equation}\label{EQ:Phi_anh}
\begin{split}
\Phi\kvkvpkvpp&=\sum_{\alpha,b}^{3,n}\sum_{\alpha',b',l'}^{3,n,N}\sum_{\alpha'',b'',l''}^{3,n,N}\delta(\pmb{\kappa}+\pmb{\kappa'}+\pmb{\kappa''}-\pmb{G})\frac{\partial^3\Phi}{\partial r_\alpha \Ob \partial r_\alpha \lbp \partial r_\alpha \lbpp}\\
&\times \frac{e \kvba e \kvbain{'}e \kvbain{''}}{\sqrt{m_{b}\omega\kv
m_{b'}\omega\kvp m_{b''}\omega\kvpp}}\EXP{i\pmb{\kappa}\cdot \pmb{r}\lO+i\pmb{\kappa'}\cdot \pmb{r}\lOin{'}+i\pmb{\kappa''}\cdot \pmb{r}\lOin{''}]}
\end{split}
\end{equation}
%
\subsection{Normal Mode Decomposition}

As is happens, the usefulness of the (time) correlation functions extends well beyond the FDT. An simple example is the Wiener-Khintchine Theorem (WKT), which relates the correlation function of a continuous stationary random process to its' spectral density. The correlation function of a time-dependent quantity (i.e: position, velocity, etc.) is defined as the average behaviour in time of said quantity \cite{mcquarrie}
%
\begin{equation}
C(t)=\lim_{T->\infty}\frac{1}{2T}\int_{-T}^{T}x(t+t')x(t')dt'
\end{equation}
%
From the ergodic hypothesis, which states that the statistical properties of a large number of observations at $N$ arbitrary times from a single system are equivalent to the statistical properties of $N$ obervations made from $N$ equivalent systems made at the same time, the correlation function can be rewritten as an ensemble average
%
\begin{equation}
C(t)=<x(t+t')x(t')>.
\end{equation}
%
Let's define $X(\omega)$ as the Fourier Transform of $x(t)$
%
\begin{equation}
X(\omega)=\int_{-\infty}^{\infty}x(t)e^{-i\omega t}dt.
\end{equation}
%
Recalling Parseval's theorem, which states that the integral of the square of a function is equal to the integral of the square of it's transform
%
\begin{equation}
\int_{-\infty}^{\infty}x^2(t)dt=\frac{1}{2\pi}\int_{-\infty}^{\infty}|X(\omega)|^2d\omega.
\end{equation}
%
Noting that $\int_{-\infty}^{\infty}x^2(t)dt=<x^2>$, let $S(\omega)$ be the spectral density of $x(t)$
%
\begin{equation}
S(\omega)=\lim_{T->\infty}\frac{1}{2T}|X(\omega)|^2.
\end{equation}
%
From the Parseval's theorem equality
%
\begin{equation}
<x^2>=\frac{1}{2\pi}\int_{-\infty}^{\infty}|X(\omega)|^2d\omega.
\end{equation}
%
To offer an intuitive interpretation of this result, take $x(t)$ to be an electric current and $<x^2>$ to be the average power dissipated as the current passes through a circuit. In this case, $X(\omega)d\omega$ will be the average power dissipated with frequencies between $\omega$ and $\omega+d\omega$. The WKT extends this result to the correlation function
%
\begin{equation}
C(t)=\frac{1}{2\pi}\int_{-\infty}^{\infty}C(\omega)e^{i\omega t}d\omega
\end{equation}
%
\begin{equation}
C(\omega)=\int_{-\infty}^{\infty}C(t)e^{-i\omega t}dt.
\end{equation}
%
Taking an example from Dove \cite{dove_introduction_1993-3}, let $x$ have only two equally probably values of $\pm 1$ with the probability of $x$ changing it's value during $dt$ of $dt/\tau$, where $\tau$ represents the average time between value changes. The correlation function is
%
\begin{equation}
C(t)=e^{\frac{-|t|}{\tau}}.
\end{equation}
%
The spectral density is then
\begin{equation}
C(\omega)=\int_{-\infty}^{\infty}e^{\frac{-|t|}{\tau}}e^{-i\omega t}dt=\frac{2\tau}{1+(\omega \tau )^2}
\end{equation}
which is a Lorentzian centred about zero frequency and $\tau$ is the half width at half-maximum (HWHM).

Recalling Eq. ~\ref{EQ:SMRTK}, the final and missing piece is the lifetime of a given mode $\tau_{p-p}\kv$ which arises as a result of the intrinsic anharmonicity of the interatomic potential and is responsible for finite thermal conductivity and thermal expansion. In the past decade, significant progress has been to computationally predict this property. Broido et al. used Density Functional Perturbation Theory (DFPT) \cite{Broido1} while Esfarjani et al. used a DFT-MD approach \cite{PhysRevB.84.085204}. The method proposed by Larkin \cite{jason_inpress} involves the calculation of the spectral energy density of the normal modes (SED-NMD). Although this approach relies on the empirical potentials of classical MD, the complete anharmonicity is considered, an advantage it posesses over other methods, like DFPT which truncate terms beyond the third order derivatives. SED-NMD is an algorithm that combines time-dependent information from molecular dynamics and the harmonic solutions from lattice dynamics to infer the phonon lifetimes From harmonic lattice dynamics, the displacement of atom $b$ in unit cell $l$ at time $t$ is represented as a superposition of waves of wavevector $\bm{\kappa}$ with amplitude $\bm{U}\kvb$
%
\begin{equation}
\bm{u}\lbt=\sum_{\bm{\kappa},\nu}\bm{U}\kvb exp(i[\bm{\kappa}\cdot\bm{r}\lb-\omega\kv t])=\frac{1}{\sqrt{Nm_j}}\sum_{\bm{\kappa},\nu}\bm{e}\kvb exp(i\bm{\kappa}\cdot\bm{r}\lb)Q\kv
\end{equation}
%
with $Q\kv$ being the normal code coordinate and $\bm{e}\kvb$ being the eigenvector determined from the eigenvalue problem $\omega^2\kv e\kv=D(\bm{\kappa})e\kv$. To rearrange for the normal mode, mulitply Equation 27 with $\bm{e}^*\kvb$ to take advantage of the orthogonality of the eigenvectors
%
\begin{equation}
\bm{e}^*\kvb\bm{u}\lbt=\frac{1}{\sqrt{Nm_j}}exp(i\bm{\kappa}\cdot\bm{r}\lb)Q\kv.
\end{equation}
%
Taking the Fourier Transform
%
\begin{equation}
\int_{-\infty}^{\infty}\bm{e}^*\kvb\bm{u}\lbt exp(-i\bm{\kappa}\cdot\bm{r}\lb)d\bm{r}=\frac{1}{\sqrt{Nm_j}}\int_{-\infty}^{\infty}Q\kv d\bm{r}
\end{equation}
%
and noting that $\int_{-\infty}^{\infty}d\bm{r}=N$, gives the expression for the normal coordinate
%
\begin{equation}
Q\lbt=\frac{1}{\sqrt{N}}\sum_{b,l}\sqrt{m_j}exp(-i\bm{\kappa}\cdot\bm{r}\lb)\bm{e}^*\kvb\cdot\bm{u}\lbt.
\end{equation}
%
The time derivative of the normal mode is
%
\begin{equation}
\dot{Q}\lbt=\frac{1}{\sqrt{N}}\sum_{b,l}\sqrt{m_j}exp(-i\bm{\kappa}\cdot\bm{r}\lb)\bm{e}^*\kvb\cdot\dot{\bm{u}}\lbt.
\end{equation}
%
The harmonic Hamiltonian of the lattice can thus be represented in terms of normal modes
%
\begin{equation}
H=\frac{1}{2}\sum_{\bm{\kappa},\nu}\dot{Q}\kv\dot{Q}^*\kv+\frac{1}{2}\sum_{\bm{\kappa},\nu}\omega^2\kv Q\kv Q^*\kv.
\end{equation}
%
The first term on the right hand side corresponds to the kinetic energy while second term corresponds to the potential energy. By taking a series of velocity samples from an equilibrium MD simulation of time interval (in signal processing terminology, this is known as lag which is symbolically represented here by $t$) an order of magnitude shorter than inverse of the highest frequency present in the system (known from solutions to the aforementioned eigenvalue problem) and using Eq. ~\ref{} to project the sampled velocities onto the eigenvectors, the autocorrelation of the normal modes can calculated by
%
\begin{equation}
C\kvt=\lim_{T->\infty}\frac{1}{T}\int_{0}^{T}Q(\bm{\kappa},\nu,t+t')Q(\bm{\kappa},\nu,t')dt'.
\end{equation}
%
The spectral energy density, from the WKT, is thus
%
\begin{equation}
C\kvw=\int_{-\infty}^{\infty}C(\bm{\kappa},\nu,t)e^{-i\omega t}dt
\end{equation}
%
which, like Equation 19, is a Lorentzian centered at $\omega_0\kv$
%
\begin{equation}
C\kvw=\frac{C_0\kv}{2}\frac{\Gamma\kv/\pi}{(\omega_0\kv-\omega)^2+\Gamma\kv}.
\end{equation}
%
The HWHM is related by anharmonic lattice dynamic theory \cite{PhysRev.128.2589} to phonon lifetime by
%
\begin{equation}
\tau_{p-p}\kv=\frac{1}{2\Gamma\kv}
\end{equation}
%
The interpretation of this relation can be understood through a qualitative argument from time-dependent perturbation theory (TDPT). Using TDPT, the anharmonic terms in the complete Hamiltonian are assumed to be small and can thus be considered to be to perturbation upon the harmonic state. The probablity amplitude carries the time-dependence in this picture. In a two-state system
%
\begin{equation}
|\Psi>=A(t)|\psi_A>+B(t)|\psi_B>
\end{equation}
%
as the amplitudes $A(t)$ and $B(t)$ vary time, so does the probability of finding the particle in state $|\psi_A>$ or $|\psi_B>$. Expressing the equivalent relation for three-phonon processes
%
\begin{equation}
|\bm{\kappa},\bm{\kappa}',\bm{\kappa}''>=A(t)|\bm{\kappa}>+B(t)|\bm{\kappa}'>+C(t)|\bm{\kappa}''>.
\end{equation}
%
The probability of a phonon scattering from $\bm{\kappa}$ to state $\bm{\kappa}'$ is governed by the relative magnitudes of the amplitudes $A(t)$ and $B(t)$ (in accordance with the selection rules of momentum and energy conservation). The broadening of these peaks corresponds to this scattering process, indicating a non-zero probability of a phonon transitioning from one state to another. The form of Eq. ~\ref{} is a consequence of Fermi's Golden Rule from TDPT.

The application of SED-NMD to compute phonon lifetimes and predict thermal conductivity assumes the validity of the phonon BTE. It remains to be determined if SED-NMD can be used to predict non-bulk phonon lifetimes (are the bulk eigenvectors accurate in non-bulk cases?)

\section {Bulk Comparison}

\begin{table}
\begin{center}
\begin{tabular*}{\textwidth}{c@{\extracolsep{\fill}}cc}
\hline\hline\noalign{\smallskip}

Method & Thermal Conductivity  \\
\noalign{\smallskip}\hline\noalign{\smallskip}
Green-Kubo & 1.2 $\pm$ 0.01\\
Direct Method & 1.7 \\
NMD & 1.2\\
ALD & 1.3\\
\hline\hline
\end{tabular*}
\end{center}
\renewcommand{\table}{Table.}
\caption{A comparison of the thermal conductivity prediction methods [$Wm^{-1}K^{-1}$].}
\label{TB:K_compare}
\end{table}

\begin{comment}
\section{Thermal Boundary Resistance}
The following section summarizes the seminal work of Swartz and Pohl on the challenges of measuring and modeling the thermal boundary resistance (TBR) \cite{RevModPhys.61.605}. TBR is defined as the ratio of the temperature discontinuity at the interface to the power per unit area flowing across that interface
\begin{equation}
TBR=\frac{T_l-T_r}{q_x}.
\end{equation}
The inverse of TBR is thermal boundary conductivity (TBC) which is defined as the ratio of heat flow per unit area to the temperature discontinuity at the interface. Kapitza was the first to report the measurement of this temperature discontinuity at the interface between helium and a solid.

To model the transmission of phonons across a solid-solid interface, Khalatnikov presented the acoustic mismatch model (AMM) and later modernized by Mazo and Onsager. The assumptions involved when using the AMM are a phonon is either transmitted across the interface or it is reflected, both sides of the interface are isotropic, the probability that a phonon is transmitted is independent of temperature and anharmonic interactions are ignored. The transmission probability is classically related to the fraction of energy that crosses the interface and acts as placeholder in the the energy balance of the interface. Within the AMM picture, phonons are interpreted as plane waves that move through a continuum; a perspective that allows one to view the transmission of phonons as anologous to the refraction of photons as they move from one medium to another. The major challenge in the AMM is to calculate the transmission probability for any incident angle and mode. Relying upon the acoustic analog to the Fresnel equations and assigning an impendance as function of the density of the material and the phonon velocity, an estimate of the transmission probability is obtained. The TBC is calculated by summing over all incoming angles and the occupancy of all phonons multiplied by the respective transmission probability.

Given the assumptions used in the AMM, it is not surprising that the results differ from experiments. The TBR between $^4$He and copper predicted by AMM was two orders of magnitude larger than experiment, a result which is not atypical for the standard AMM. In an attempt to resolve this discrepancy, Swartz proposed the Diffuse Mismatch Model (DMM) where, unlike the AMM in which phonons are specularly transmitted or reflected, all phonons are diffusely scattered, forward or backward, at the interface. Scattering is assumed to destroy any information about the phonon prior to scattering, where the possible modes available following scattering are determined by an energy balance and the phonon density of states. DMM tends to overestimate the amount of scattering at an interface. With such idealizations, AMM and DMM calculations of TBR are generally an order of magnitude different from experiment at temperatures above 30 K \cite{landrythesis}, but are useful to estimate an upper and lower limit (not necessarily in the respective order) to the TBR of a given interface system.

The incorporation of the inelastic scattering of phonons near and at the interface  into thermal transport models is required in order to improve predictions of TBR. Landry used anharmonic LD and the phonon BTE to determine the TBR across Si/Ge and Si/heavy-Si interfaces in non-equilibrium high-temperature conditions \cite{landrythesis}. The predicted TBRs were found to be 40-60\% less than the TBR predictions from MD with the direct method, where phonon scattering is inherently included This was one of the first results to conclusively demonstrate the importance of phonon scattering upon TBR. However, the effect upon phonon properties near an interface could not be elucidated from such an approach (Landry attributed the discrepancy in TBR predictions to deviations from the bulk phonon density of states, but did not examine anharmonic changes). The following section discusses the challenges of studying spatially-dependent phonon properties.
\end{comment}



\chapter{Interface Study}

\section{Thermal Boundary Resistance}
The following section summarizes the seminal work of Swartz and Pohl on the challenges of measuring and modeling the thermal boundary resistance (TBR) \cite{RevModPhys.61.605}. TBR is defined as the ratio of the temperature discontinuity at the interface to the power per unit area flowing across that interface
\begin{equation}
TBR=\frac{T_l-T_r}{q_x}.
\end{equation}
The inverse of TBR is thermal boundary conductivity (TBC) which is defined as the ratio of heat flow per unit area to the temperature discontinuity at the interface. Kapitza was the first to report the measurement of this temperature discontinuity at the interface between helium and a solid.

To model the transmission of phonons across a solid-solid interface, Khalatnikov presented the acoustic mismatch model (AMM) and later modernized by Mazo and Onsager. The assumptions involved when using the AMM are a phonon is either transmitted across the interface or it is reflected, both sides of the interface are isotropic, the probability that a phonon is transmitted is independent of temperature and anharmonic interactions are ignored. The transmission probability is classically related to the fraction of energy that crosses the interface and acts as placeholder in the the energy balance of the interface. Within the AMM picture, phonons are interpreted as plane waves that move through a continuum; a perspective that allows one to view the transmission of phonons as anologous to the refraction of photons as they move from one medium to another. The major challenge in the AMM is to calculate the transmission probability for any incident angle and mode. Relying upon the acoustic analog to the Fresnel equations and assigning an impendance as function of the density of the material and the phonon velocity, an estimate of the transmission probability is obtained. The TBC is calculated by summing over all incoming angles and the occupancy of all phonons multiplied by the respective transmission probability.

Given the assumptions used in the AMM, it is not surprising that the results differ from experiments. The TBR between $^4$He and copper predicted by AMM was two orders of magnitude larger than experiment, a result which is not atypical for the standard AMM. In an attempt to resolve this discrepancy, Swartz proposed the Diffuse Mismatch Model (DMM) where, unlike the AMM in which phonons are specularly transmitted or reflected, all phonons are diffusely scattered, forward or backward, at the interface. Scattering is assumed to destroy any information about the phonon prior to scattering, where the possible modes available following scattering are determined by an energy balance and the phonon density of states. DMM tends to overestimate the amount of scattering at an interface. With such idealizations, AMM and DMM calculations of TBR are generally an order of magnitude different from experiment at temperatures above 30 K \cite{landrythesis}, but are useful to estimate an upper and lower limit (not necessarily in the respective order) to the TBR of a given interface system.

The incorporation of the inelastic scattering of phonons near and at the interface  into thermal transport models is required in order to improve predictions of TBR. Landry used anharmonic LD and the phonon BTE to determine the TBR across Si/Ge and Si/heavy-Si interfaces in non-equilibrium high-temperature conditions \cite{landrythesis}. The predicted TBRs were found to be 40-60\% less than the TBR predictions from MD with the direct method, where phonon scattering is inherently included This was one of the first results to conclusively demonstrate the importance of phonon scattering upon TBR. However, the effect upon phonon properties near an interface could not be elucidated from such an approach (Landry attributed the discrepancy in TBR predictions to deviations from the bulk phonon density of states, but did not examine anharmonic changes). The following section discusses the challenges of studying spatially-dependent phonon properties.

\section{NMD near an interface}

NMD was used in an attempt to observe the effect of an interface upon phonon properties, namely phonon lifetimes. Here, an interface is defined by enforcing a mass difference in a LJ argon system as seen in Fig. ~\ref{interface_domain}. In order to ensure the statistical significance of the results, an averaging scheme was needed. For each case, five independent MD simulations were conducted, each with a different initial velocity seed. In each MD simulation, 16 sets of atomic velocities were produced. Each set of velocities consisted of 2048 subsets of atomic velocities; the lag between subsets was 32 LJ time units (in other words, velocities were sampled every 32 LJ time units for a total of 2048 samples). The power of two formulation was chosen for the sake of the fast Fourier Transform. $\dot{Q}(\bm{\kappa},\nu,t)$ was calculated for each individual velocity subset (Eq.~\ref{}) and then used as input for the correlation function (Eq.~\ref{}) and SED (Equation \ref{}) of the set. Finally the SED is averaged over the 16 sets and 5 seeds. This procedure was performed on three cases: (A) a 4 unit cell by 4 unit cell by 4 unit cell (hereby referred to as $4\times4\times4$) domain of LJ argon in equilibrium at 20 K with periodic boundary conditions (B) a $32\times4\times4$ domain of LJ argon at 20 K with periodic boundary conditions and (C) $32\times4\times4$ domain of Lennard-Jones argon at 20 K, where one half ($16\times4\times4$) is set to the unit mass of argon and the other half is set to three times the unit mass of argon, with periodic boundary conditions. Cases B and C are divided into $4\times4\times4$ blocks for the post-processing steps to match the phonon modes present in Case A (see Fig. ~\ref{}), in an attempt to offer a mode by mode comparison. All MD simulations were performed at 20 K with periodic boundary conditions in all directions.
%%%
\begin{figure}%[ht!]
\begin{center}
\scalebox{0.75}{ \includegraphics{supcell_ai.eps}}
\renewcommand{\figure}{Fig.}
\caption{MD domain for interface study of $32\times4\times4$ FCC argon at 20K with 2048 atoms. Larger atomic radii represents the heavier mass.}
\label{fig:interface_domain}
\end{center}
\end{figure}
%%%
As the plane of the interface has a normal in the $x$ direction, it is reasonable to expect wavevectors containing some $x$ component to be affected. For simplicity, the modes at $\bm{\kappa}=[1,0,0]$ in the BZ are examined. 
%%%
\renewcommand{\topfraction}{1.0}
\begin{figure*}%[t]
\begin{center}
\scalebox{1}{ \includegraphics{sed_int.eps}}
\renewcommand{\figure}{Fig.}
\caption{Plots of example power spectrums.}
\label{fig:sed}
\end{center}
\end{figure*}
%%%
At first glance, it is clear that the SED of Cases B and C differ from Case A. The essence of the difference between Cases B and C and Case A lies not in the physics of the phonons, but in the model of their description. Ultimately, the idea of dividing the $32\times4\times4$ domains into $4\times4\times4$ blocks and performing NMD on these blocks to observe the change in the phonon lifetimes as a function of spatial position relative to the interface proved to be an ineffective approach.

The precise reason for these results is attributed to the mathematical nature of the problem. A sketch of the reason is offered here (a formal argument is presented in Appendix~\ref{appendix:b}). First, the definition of $Q(\bm{\kappa},\nu)$ requires the summation over all possible $\bm{\kappa},\nu$ of the entire $32\times4\times4$ domain. The division into blocks discounts the entire summation to only include available $\bm{\kappa},\nu$ of the $4\times4\times4$ block:
\begin{eqnarray}
\begin{split}
\dot{Q}(\bm{\kappa},\nu,t)-\frac{1}{\sqrt{N}}\sum_{28x4x4}\sqrt{m_j}exp(-i\bm{\kappa}\cdot\bm{r}(jl))\bm{e}^*(j,\bm{\kappa},\nu)\cdot\dot{\bm{u}}(jl,t)&=\\\frac{1}{\sqrt{N}}\sum_{4x4x4}\sqrt{m_j}exp(-i\bm{\kappa}\cdot\bm{r}(jl))\bm{e}^*(j,\bm{\kappa},\nu)\cdot\dot{\bm{u}}(jl,t).
\end{split}
\end{eqnarray}
Neglecting the modes that are uniqe to the complete $32\times4\times4$ domain, leads to an erroneous coordinate transformation from real space to $\pmb{\kappa}$-space because of the incompleteness of the solution basis. In other words, without using all the possible modes of the system to describe the motion of an individual atom, the corresponding normal mode cannot be inferred. The reverse is equally true.

Futhermore, the type of statistical ensemble of each individual block is difficult to ascertain. The entire domain is fixed to be NVE, but the energy of a block may not be fixed with time (in other words, the Hamiltonian of a block is not time-independent). At any given instant the energy of one block may be more or less than one of its neighbours, but the energy of all the blocks together remains constant.

This exercise revisits the struggle to handle any deviation from the perfectly periodic bulk crystal lattice. By relying on the eigenvectors and frequencies obtained from lattice dynamics, it is not obvious how to incorporate aperiodicities into this approach.



\chapter{Superlattice Study}\label{CHP:SL}

\section{Background}

Superlattices are nanostructures built from periodic alternating layers of dissimilar materials. Semiconductor superlattices, where phonons dominate the thermal transport, offer potential benefits in thermoelectric energy conversion applications because of the ability to tune their thermal conductivities by controlling the thicknesses of the layers (i.e., the secondary periodicity) without significantly affecting the electronic transport \cite{broido1995effect,balandin2003mechanism,kim2006thermal}. To design a superlattice with a tailored thermal conductivity, a rigorous examination of the interplay between the superlattice period thickness and interfacial mixing, present in any experimental sample, is necessary. This need provides the impetus for an analysis to elucidate the effects of the secondary periodicity and interfacial mixing on the properties of individual phonon modes. 

Thermal transport in superlattices has been studied using molecular dynamics (MD) simulations and the Boltzmann transport equation. Previous MD studies used the equilibrium Green-Kubo (GK) \cite {PhysRevB.85.195302,PhysRevB.77.184302} technique or the non-equilibrium \cite{PhysRevB.77.184302,PhysRevB.79.214307,PhysRevB.72.174302,PhysRevB.79.075316} direct method to predict thermal conductivity. Bottom-up studies using the BTE relied upon the validity of bulk phonon properties in each layer \cite{walkauskas:2579,chen:220} and approximations for the specularity and conductance of the internal interfaces \cite{PhysRevB.57.14958}. While these techniques can predict trends in thermal conductivity versus period length, the effects of the secondary periodicity and interfacial mixing on phonon properties cannot be directly obtained.

The effect of interfacial mixing on the thermal conductivity (but not on individual phonon modes) of Si/Ge superlattices was examined by Landry and McGaughey \cite{PhysRevB.79.075316}. They showed that for perfect Si/Ge superlattices thermal conductivity decreased with increasing period length before leveling out, while for superlattices with interfacial mixing, the thermal conductivity (always lower than the corresponding perfect case) increased with period length before leveling out. Savic \textit{et al.} used Monte Carlo integration to predict the phonon properties in perfect Si/Ge superlattices (i.e., inclusion of the secondary periodicity without interfacial mixing) and noted that theoretical results over- predicted experimental data \cite{savic:073113}. Recent work by Garg \textit{et al.} used density functional perturbation theory (DFPT) to examine phonon properties in perfect Si/Ge superlattice structures (i.e., inclusion of the secondary periodicity without interfacial mixing). They report a significant discrepancy between theoretically-predicted and experimentally-measured cross-plane thermal conductivities, which they attributed to the exclusion of mass defect scattering in the DFPT calculation but expected to be important in experimental samples \cite{doi:10.1021/nl202186y}. With this result in mind, Luckyanova \textit{et al.} \cite{Luckyanova16112012} adopted Tamura elastic mass defect scattering theory \cite{tamura_isotope_1983} to modify the DFPT predicted lifetimes through the Matthiesen rule for Si/Ge superlattices (i.e., inclusion of the secondary periodicity and interfacial mixing). Garg and Chen emphasized the importance of including interfacial mixing using DFPT and Tamura theory by observing a tenfold reduction in cross-plane thermal conductivity from perfect (unmixed) to mixed Si/Ge superlattices \cite{PhysRevB.87.140302}.
%Recent work by Garg et al. used Density Functional Perturbation Theory (DFPT) to examine phonon properties in Si/Ge superlattice structures to report a significant discrepancy between theoretically predicted and experimentally measured cross-plane thermal conductivities which was attributed to the exclusion of mass defect scattering in the DFPT calculation but expected to be present in experimental samples. \cite{doi:10.1021/nl202186y} This conclusion was echoed by Savic et al. using a Monte Carlo approach to the BTE for Si/Ge superlattices. \cite{savic:073113} The effect of defects on thermal conductivity was demonstrated by Landry and McGaughey, who showed that for perfect Si/Ge superlattices thermal conductivity decreased with increasing period length before eventually leveling out while for superlattices with interfacial mixing, the thermal conductivity increased with period length before eventually leveling out. \cite{PhysRevB.79.075316} With this in mind, Luckyanova et al. \cite{Luckyanova16112012} adopted Tamura elastic mass defect scattering theory \cite{tamura_isotope_1983} to modify the DFPT predicted lifetimes through the Matthiesen rule.
%Mode-by-mode studies are computationally expensive; limiting Broido et al. to a range of short period lengths ($8\times 8$) for Si/Ge.\cite {PhysRevB.70.081310} For similar reasons, the phonon properties for short period superlattices obtained from DFPT were presumed to hold for larger period superlattices \cite{Luckyanova16112012, doi:10.1021/nl202186y} and the phonon lifetimes of low frequency modes used in the Monte Carlo BTE approach are fitted to a power law. \cite{savic:073113}

In this paper, we explore the direct relationship between the secondary periodicity and interfacial mixing on phonon properties. The vibrational modes which emerge by using the phonon dispersion based on the secondary periodicity of the layers are hereafter referred to has superlattice phonons. Molecular dynamics-based normal mode decomposition (NMD) is used to predict the full spectrum of phonon properties in unstrained superlattices with perfect and mixed interfaces. The rest of the paper is organized as follows. In Section~\ref{SEC:modeling}, the superlattice geometry is defined and the NMD algorithm is reviewed. In Section~\ref{SEC:results}, superlattice phonon dispersion, phonon lifetimes, and thermal conductivities predicted from GK, NMD, and Tamura theory are presented. In Section~\ref{SEC:sl_phon}, the results are put in context with the concept of phonon coherence.

%The crucial ingredient is to use the eigenvectors (polarization vectors) of the perfect superlattice to obtain lifetimes for both perfect and mixed superlattices. 

\section{Modeling Framework}\label{SEC:modeling}
\subsection{Superlattice structure and interactions}\label{SEC:sl_struc}
%%%
The superlattices are built by placing atoms on a face-centered cubic lattice, with the two species only differentiated by their masses. The atomic interactions are modeled using the Lennard-Jones (LJ) potential for argon with an energy scale of $1.67\times10^{-21}$ J, a length scale, $\sigma$, of $3.4\times10^{-10}$ m, a mass scale, $m$, of $6.63\times10^{-26}$ kg, and are cutoff and shifted at a radius of $2.5\sigma$. The lighter species has a mass of $m$ and the heavier species has a mass of $3m$. The temperature of all simulations is 20 K, for which the zero-pressure lattice constant, $a$, is 5.315 $\AA$ \cite{mcgaugheythesis}. We present results in  dimensionless LJ units unless otherwise noted. 
%($E^*=E/\epsilon$, $T^*=Tk_B/\epsilon$, $\omega^*=\omega\sqrt{\sigma^2m_a/\epsilon}$, $k^*=km^{0.5}_a\sigma^2/(\epsilon^{0.5}k_B)$)

Each superlattice is identified by its unit cell, which consists of $L/2$ conventional four-atom unit cells of each species. The unit cell therefore contains $4L$ atoms. As shown in Fig.~\ref{fig:md_domain}(a), one period of a $4\times4$ superlattice has eight monolayers (four of each species). The Brouillin zone is a rectangular prism with boundaries at $2\pi/(La)$ in the cross-plane direction and $2\pi/a$ in the in-plane directions. We consider $2\times2$, $4\times4$, $8\times8$, and $14\times14$ superlattices.
%%%
\begin{figure}[t!]
\begin{center}
\scalebox{0.5}{ \includegraphics{4p_ai.eps}}
\renewcommand{\figure}{Fig.}
\caption{Atomic representation of a $4\times4$ superlattice for (a) perfect and (b) 80/20 interfacial mixing cases. Orange atoms have mass  $m$ and green atoms have mass $3m$.}
\label{fig:md_domain}
\end{center}
\end{figure}
%%%

Interfacial mixing is introduced to a superlattice by flipping the masses of randomly selected atoms in the monolayers adjacent the interfaces until the desired concentrations are reached \cite{PhysRevB.79.075316}. A $4\times4$ superlattice with 80/20 interfacial mixing (the notation corresponds to the concentration of original/foreign species within the monolayers adjacent to the interface) is shown in Fig.~\ref{fig:md_domain}(b).

\subsection{Thermal conductivity prediction}\label{SEC:methods}

As in some previous superlattice studies \cite{Luckyanova16112012,doi:10.1021/nl202186y}, a solution to the phonon BTE under the single mode relaxation time approximation \cite{ziman_electrons_2001}, is used to predict the thermal conductivity, $k$, in the $\alpha$-direction from
%%%
\begin{equation}\label{EQ:M:conductivity}
\begin{split}
k_{\mathbf{\alpha}}=&\sum_{\nu,\pmb{\kappa}}^{12L,N} c_{ph}\kv
v^{2}_{g,\mathbf{\alpha}}\kv \tau\kv.
\end{split}
\end{equation}
%%%
Here, $c_{ph}\kv$ is the volumetric specific heat, $v_{g,\mathbf{\alpha}}\kv$ is the component of the group velocity in the $\alpha$-direction, and $\tau\kv$ is the lifetime of the phonon mode with wavevector $\pmb{\kappa}$ and polarization branch denoted by $\nu$. The summation is over the total number of polarization branches, $12L$, and the number of unit cells in the simulation, $N$. A quantity of interest for nanostructure design purposes \cite{PhysRevB.87.035437} is the phonon mean free path (MFP), $\Lambda\kv$, defined as the average distance travelled between scattering events \cite{ziman_electrons_2001},
%%%
\begin{equation}\label{EQ:M:phonon_mfp}
\begin{split}
\Lambda\kv &= |\pmb{\mathrm{v}}_{g}\kv | \tau\kv,
\end{split}
\end{equation}
%%%
where $\pmb{\mathrm{v}}_{g}\kv$ is the group velocity vector. To obtain the required inputs for Eqs.~(\ref{EQ:M:conductivity}) and~(\ref{EQ:M:phonon_mfp}), we follow the NMD procedure outlined by McGaughey and Kaviany \cite{PhysRevB.71.184305}, Turney \textit{et al.} \cite {PhysRevB.79.064301}, and Larkin \textit{et al.} \cite{jason_inpress}, in which atomic velocities obtained from MD simulation are projected onto the normal mode eigenvectors obtained from harmonic lattice dynamics calculations. The mode-dependent specific heat is set to be $k_\mathrm{B}/V$, where $V$ is the volume of the MD domain and $k_\mathrm{B}$ is the Boltzmann constant, because MD simulations are classical and obey Maxwell-Boltzmann statistics. As temperature increases, anharmonicity of the potential energy causes the specific heat to deviate from $k_\mathrm{B}/V$, but the effect is small for LJ systems at the studied temperature of 20 K \cite{PhysRevB.71.184305}. 

The unit cells for the perfect superlattices, as depicted in Fig.~\ref{fig:md_domain}(a), are used as inputs to the harmonic lattice dynamics calculations which are performed using GULP \cite{GULP} to obtain harmonic frequencies, $\omega_H \kv$, and eigenvectors, $ \pmb{\mathrm{e}} \kv$, at the allowed wavevectors, which are specified from
%%%
\begin{equation}\label{EQ:NMD:allowdkpt}
\pmb{\kappa} = \sum_{\alpha=1}^3 \pmb{\mathrm{b}}_{\alpha} \frac{n_{\alpha}}{N_{\alpha}}.
\end{equation}
%%%
Here, $\pmb{\mathrm{b}}_\alpha$ are the cubically orthogonal reciprocal lattice vectors and, to ensure that all wavevectors are in the first Brouillin zone, $ -\frac{N_\alpha}{2} < n_\alpha \le \frac {N_\alpha}{2}$, where $n_\alpha$ are integers and $N_\alpha$ are constant even integers corresponding to the number of unit cells in the $\alpha$ direction in the MD domain.
\begin{comment}
Group velocities are calculated using finite differencing about a given frequency from \cite{ziman_electrons_2001}
%%%
\begin{equation}\label{EQ:NMD:vg}
\begin{split}
\pmb{\mathrm{v}}_{g}\kv=\frac{\partial \omega \kv}{\partial \pmb{\kappa}}.
\end{split}
\end{equation}
%%%
The eigenvectors, from HLD, and the atomic velocities, from MD simulations performed using LAMMPS\cite{LAMMPS}, are are used as inputs to obtain the trajectory of the time derivative at time, $t$, of the normal mode coordinate, $\dot{q}\kvt{}{}{}$, at time $t$ from
%%%
\begin{equation}\label{EQ:NMD:qdot}
\begin{split}
\dot{q}\kvt{}{}{}=&\SUM{0}{}\sqrt{\frac{m_b}{N}}\dot{u}_{\alpha}\lbt e^*\kvba\EXP{i\pmb{\kappa}\cdot\mathbf{r}_0\ab{l}{0}}.
\end{split}
\end{equation}
%%%
In Eq.~(\ref{EQ:NMD:qdot}), $\dot{u}_{\alpha}\lbt$ is the component of velocity of atom $b$ in the $l$th unit cell with equilibrium position $\mathbf{r}_0\ab{l}{0}$ in direction $\alpha$ and $e^*\kvba$ denotes the complex conjugate of the $\alpha$-component for atom $b$ of the eigenvector for mode  ~$\kv$. 
\end{comment}
While the harmonic lattice dynamics calculations were performed using the unit cells of perfect superlattices, we use the same set of eigenvectors to obtain the time derivative of the normal mode coordinate for both perfect and mixed superlattices. The effects of mixing are thus captured through the differences in the atomic velocities between perfect and mixed MD domains. The validity of this assumption for the mixed cases (i.e. projecting onto an approximation of the normal mode) will be assessed in Section~\ref{SEC:results}.
\begin{comment}
By taking the Fourier transform of the autocorrelation of Eq. (\ref{EQ:NMD:qdot}), the mode power spectrum is obtained: \cite{dove_introduction_1993-3}
%%%
\begin{equation}\label{EQ:NMD:SED}
\begin{split}
T\kvw=&\lim_{\tau_0\rightarrow\infty}\frac{1}{2\tau_0}\left|\frac{1}{\sqrt{2\pi}}\int_{0}^{\tau_0}\dot{q}\kvt\exp(-i\omega t)dt\right|^2.
\end{split}
\end{equation}
%%% 
\end{comment}
The MD simulation time step is 4.285 fs. Equilibration is achieved by velocity rescaling for $10^5$ timesteps followed by a \textit{NVE} (constant mass, volume and total energy) ensemble for $2.5 \times10^5$ timesteps. The Fourier transform sampling window, $\tau_0$, was set to depend upon the superlattice system and the mode frequency. The number of timesteps for the Fourier transform sampling window and total simulation time are given in Table~\ref{TB:MD_time}. The lag between velocity samples was $2^5$ time steps, which is sufficient to capture the dynamics of the highest-frequency modes. The power spectrum was averaged over the Fourier transform sampling windows and over the number of independent MD simulations, with the initial atomic velocities sampled from a Gaussian distribution using a random seed (see Table~\ref{TB:MD_time}). Further averaging was conducted by imposing the symmetry of the irreducible Brouillin zone.

%set to $2^{16}$ time steps for all modes in the $2 \times 2$ and $4 \times 4$ superlattice and for modes with frequencies greater than one in the $8\times 8$ and $14 \times 14$ superlattices, with a total simulation time of $2^{20}$ for five independent MD simulations with different initial velocity seeds. The Fourier transform sampling window was set to $2^{20}$ and $2^{22}$ time steps for modes with frequencies less than one for $8\times 8$ and $14 \times 14$ superlattices, with total simulation time of $2^{20}$ and $2^{22}$ for ten independent MD simulations with different initial velocity seeds.
%%%
\begin{table*}
\begin{center}
\begin{tabular*}{\textwidth}{c@{\extracolsep{\fill}}ccccc}
\hline\hline\noalign{\smallskip}
&\multicolumn{4}{c}{Superlattice} \\
\cline{2-5}\noalign{\smallskip}
\hspace{1cm} & $2\times2$ & $4\times4$ & $8\times8$ & $14\times14$  \\
\noalign{\smallskip}\hline\noalign{\smallskip}
FFT window, $\tau_0$ ($\omega_H \kv \geq 1$/$\omega_H \kv < 1$) & $2^{16}/2^{16}$ & $2^{16}/2^{16}$ & $2^{16}$/$2^{20}$ &$ 2^{16}$/$2^{22}$\\
Total MD timesteps ( $\omega_H \kv \geq 1$/$\omega_H \kv <1$) & $2^{20}/2^{20}$ &  $2^{20}/2^{20}$ & $2^{20}$/$2^{20}$  & $2^{20}$/$2^{22}$\\
Number of Seeds ( $\omega_H \kv \geq 1$/$\omega_H \kv<1$)& 5/5 &  5/5 & 5/10  &  5/10\\
\hline\hline
\end{tabular*}
\end{center}
\renewcommand{\table}{Table.}
\caption[Number of timesteps in the Fourier sampling windows, total number of MD timesteps for each superlattice system, and the total number of independent MD simulations.]{Number of timesteps in the Fourier sampling windows, total number of MD timesteps for each superlattice system, and the total number of independent MD simulations. The use of $\omega_H\kv=1$ as the transition frequency was found to be suitable in order to obtain convergence for lifetime estimates and to satisfy the $\Gamma\kv \ll \omega_H\kv$ condition.}
\label{TB:MD_time}
\end{table*}
%%%
\begin{comment}
In accordance with anharmonic theory,\cite{maradudin_scattering_1962} the power spectrum, given by Eq.~(\ref{EQ:NMD:SED}) can be approximated to be a Lorentzian function, centered at $\omega_A\kv$, which is shifted from the $\omega_H\kv$,  with a full width at half maximum $\Gamma\kv$ of the form 
%%%
\begin{equation}\label{EQ:NMD:LOR}
T\kvw \approx C_0\kv\frac{\Gamma\kv/\pi}{[\omega_0\kv-\omega]^2+\Gamma^2\kv},
\end{equation}
%%%
when $\Gamma\kv \ll \omega_H\kv$. The phonon lifetime from\cite {PhysRevB.81.081411}
%%%
\begin{equation}\label{EQ:lifetime}
\tau\kv=\frac{1}{2\Gamma\kv}.
\end{equation}
%%%
\end{comment}

Fitting Eq.~(\ref{EQ:NMD:LOR}) to Eq.~(\ref{EQ:NMD:SED}) was done by considering points within three orders of magnitude of the maximum value. The initial guess for $\Gamma\kv$ was 0.01 and for $\omega_A\kv$, the frequency at the maximum value of $T\kvw$ was used. 

The GK method, which makes no assumptions about the nature of the thermal transport, was also used to predict the thermal conductivity for both perfect and mixed superlattices. Landry \textit{et al.} previously applied the GK method to LJ superlattices, finding agreement with predictions from non-equilibrium MD simulations and an application of the Fourier Law \cite{PhysRevB.79.075316}.
Ten independent MD simulations were performed for each superlattice. For the $2 \times 2$ and $4 \times 4$ superlattices, the total  simulation length was $10^6$ timesteps with a correlation window of $5\times 10^4$ timesteps.  For the $8 \times 8$ and $14 \times 14$ superlattices, the total  simulation length was $10^6$ timesteps with a correlation window of $10^5$ timesteps. In order to minimize the uncertainty in the GK predictions, the converged value of the thermal conductivity was specified using the first-avalanche method described by Chen \textit{et al.} \cite{Chen20102392}

To provide additional context between thermal conductivity prediction methods, anharmonic lattice dynamics (ALD) was performed using the unit cells for unmixed systems described in Sec.~\ref{SEC:sl_struc}. Using the third order force constants of the LJ potential and perturbation theory, the phonon lifetimes are extracted and thermal conductivity is estimated using Eq.~(\ref{EQ:M:conductivity}). %The details of ALD can be found elsewhere \cite{PhysRevB.79.064301}.

\section{Results}\label{SEC:results}
\subsection{Dispersion and Power Spectra}

Phonon dispersion curves for the perfect $4\times4$ superlattice are shown in Figs.~\ref{fig:dispersion}(a)-\ref{fig:dispersion}(c). Fig.~\ref{fig:dispersion}(a) corresponds to the cross-plane direction, Fig.~\ref{fig:dispersion}(c) corresponds to the in-plane direction and Fig.~\ref{fig:dispersion}(b) corresponds to the $[1 1 1]$ direction. Frequency gaps emerge at the Brouillin zone boundaries as a consequence of branch folding \cite{PhysRevB.38.1427,PhysRevB.60.2627}. The flat branches for frequencies greater than 15 in Fig.~\ref{fig:dispersion}(a) indicate low cross-plane group velocities. The branches in Fig.~\ref{fig:dispersion}(b) and~\ref{fig:dispersion}(c), on the other hand, vary strongly with frequency at most wavevectors. From Eq.~(\ref{EQ:M:conductivity}), we note that differences between in-plane and cross-plane components of group velocity are solely responsible for the directional dependence of the thermal conductivity.  Dispersion curves for other superlattices show similar features, with more branches and decreasing length of the cross-plane dimension of the Brouillin zone as the period length increases (the length of the Brouillin zone in the in-plane remains constant at $2\pi/a$).

The superlattice density of states, plotted as the solid blue line in Fig.~\ref{fig:dispersion}(d), shares the $\omega^2$ dependence at low frequencies with that of the bulk of the heavier species (dotted green line). The inverse participation ratio, defined as \cite{PhysRevB.70.235214}
%%%
\begin{equation}\label{EQ:P_ratio}
\begin{split}
\frac{1}{p\kv}=\sum_{b,\alpha}e\kvba^4,
\end{split}
\end{equation}
%%%
quantifies the number of atoms that participate in a given mode shape. For completely delocalized modes, $1/p\sim 1/(4L)$, and for spatially localized modes, $1/p\sim 1$. The results in Fig.~\ref{fig:dispersion}(e), where the vertical line corresponds to $1/(4L)$, indicate that there is no spatial localization in this superlattice. The frequency-dependence of the participation ratio and density of states were not found to vary with superlattice period length.
%%%
\renewcommand{\topfraction}{0.7}
\begin{figure*}%[H]
\begin{center}
\scalebox{1.1}{ \includegraphics{4p_dis_dos_pnum.eps}}
\renewcommand{\figure}{Fig.}
\caption{(a,b,c) Dispersion, (d) density of states and (e) inverse participation ratio for a $4\times4$ superlattice. Labeled gray squares represent select modes for Fig.~\ref{fig:sed}. X-$\Gamma$ corresponds to wavevectors along the cross-plane ([1 0 0]) direction, $\Gamma$-A corresponds to wavevectors along the [1 1 1] direction and Z-$\Gamma$ corresponds to wavevectors along the in-plane ([0 1 0]) direction. Orange lines correspond the bulk of the lighter species and green lines correspond to the bulk of the heavier species.}
\label{fig:dispersion}
\end{center}
\end{figure*}
%%%

The power spectra, Eq. (\ref{EQ:NMD:SED}), for the nine labeled points in Figs.~\ref{fig:dispersion}(a)-\ref{fig:dispersion}(c) are plotted in Fig.~\ref{fig:sed} for both perfect and mixed (80/20 and 60/40) superlattices. While all peaks appear to be Lorentzian centered about a single frequency, there are minor signatures at other frequencies. For perfect superlattices, the amplitude of these signatures are two orders magnitudes smaller than the main peak; the largest being found for mode (A). We attribute these secondary signatures to the assumption that the normal modes of the harmonic system are representative of the vibrational modes of the true anharmonic system. In mixed superlattices, the intensity of the signatures becomes amplified (B, E, H), an indication of the mutual implications of elastic scattering from point defects (random masses with linear springs) and anharmonicity (ordered masses with non-linear springs) on phonon scattering \cite{RevModPhys.53.175}. Modes around a frequency of 12 (B, E, H) experience the largest disruptions for all dispersion directions. We attribute this to be a result of the large number of modes with similar frequencies, as predicted by elastic scattering theory, [see Fig.~\ref{fig:dispersion}(d)] available for these modes to interact and scatter with \cite{tamura_isotope_1983}. 

Fitting a Lorentzian function [Eq. (\ref{EQ:NMD:LOR})] to obtain the reported lifetimes was deemed suitable for the 80/20 systems since the coefficient of determination value \cite{Cowpe20081066} for the most affected modes was 0.9. For the 60/40 systems, however, the peaks were disrupted to such a point where fitting the Lorentzian function was no longer suitable [see mode (B)]. This disruption is evidence that perfect superlattice modes are not always good descriptions of modes in the mixed superlattices. Since the same number of modes are present in both perfect and mixed systems, the superlattice phonons that emerge from the secondary periodicity are effectively disrupted as the crystal symmetry is broken by the interfacial mixing. For the remainder of this work, the 80/20 superlattices are used to discuss the effects of interfacial mixing on phonon properties.
%Larkin observed similar features in alloy systems.\cite{jason2013vc}
%The superlattice dispersion allows one to capture the effects of the secondary periodicity on the group velocities and lifetimes. Interfacial mixing, however, disrupts this secondary periodicity. Formally this disruption means that the perfect superlattice normal modes are not exact descriptions of the mixed superlattices. Practically, however, the results shown in Fig.~\ref{fig:sed} suggest that the perfect superlattice normal modes are a good basis for describing the behavior of the mixed superlattices. %In other words, the superlattice phonons that emerge from the secondary periodicity are effectively \textit{transformed} by the interfacial mixing. 
%%%
\renewcommand{\topfraction}{1.0}
\begin{figure*}%[t]
\begin{center}
\scalebox{1}{ \includegraphics{sed.eps}}
\renewcommand{\figure}{Fig.}
\caption{Power spectra for selected modes of the $4\times 4$ perfect and mixed superlattices [indicated by the labeled gray square markers in Figs.~\ref{fig:dispersion}(a)-\ref{fig:dispersion}(c)]. Dark blue corresponds to a perfect superlattice, red corresponds to mixing of 80/20, and light blue corresponds to mixing of 60/40. Reported lifetimes calculated from the fitting of the Lorentzian functions (not shown) are also included. By removing a single MD seed, the average uncertainty in the fitting was determined to be 7.5\%.}
\label{fig:sed}
\end{center}
\end{figure*}
%%%

%The consequences of using same set of eigenvectors were used in the NMD procedure for mixed and non-mixed $N\times N$ is observed in Figure~\ref{fig:sed}, where for modes at low and high frequency, the peaks remain well-defined, but some intermediate frequency modes become noticeably perturbed.  
%%%
%\begin{figure}%[H]
%\begin{center}
%\scalebox{0.8}{ \includegraphics{/home/schuberm/Dropbox/git/plots.nogit/images/NMD_v_ALD.eps}}
%\renewcommand{\figure}{Fig.}
%\caption{Comparison between lifetimes from NMD and ALD in a $4\times4$ superlattice without interfacial mixing.}
%\label{FIG:NMD_v_ALD}
%\end{center}
%\end{figure}
%%%
%Confirm the validity of ALD

%Figure~\ref{FIG:NMD_v_ALD} shows no systematic bias between ALD and NMD for shorter lifetimes, with some systematic scatter at the longer lifetimes towards ALD. In general, ALD lifetimes are expected to be larger than NMD lifetimes because ALD neglects the contribution from $n$ order phonon processes\cite{PhysRevB.79.064301,esfarjani2011heat}, where $n$ is greater than 3.

\subsection{Lifetimes}

The phonon lifetimes as a function of $\omega_H \kv$ are plotted in Fig.~\ref{FIG:lifetime}. The lifetimes for all superlattices exhibit $\omega^{-2}$ scaling at low frequencies, consistent with theoretical predictions for modes in the Debye regime where the density of states scales as $\omega^{2}$ as seen in Fig.~\ref{fig:dispersion}(d) \cite{Klemens_Thermal_1951}. As the period length increases, we maintain the same number of unit cells, such that the total number of atoms increase. Consequently, the minimum frequency decreases and thus the longest lifetime increases with increasing period length. For $4\times4$, $8\times8$ and $14\times14$ superlattices, the perfect systems have two distinct trends that terminate at the maximum frequencies of the corresponding bulk systems (vertical lines in Fig.~\ref{FIG:lifetime}). The magnitudes of the lifetimes do not vary significantly across perfect superlattices and are comparable to the lifetimes at the corresponding bulk frequencies.

Under the Debye approximation, a $\omega^{-4}$ lifetime scaling is predicted due to elastic phonon-point defect scattering \cite{PhysRev.140.A1812,klemens_scattering_1955-3, klemens_thermal_1957-2}. As interspecies mixing is introduced to the superlattices, a $\omega^{-4}$ scaling is observed at intermediate frequencies for the $2\times2$ and $4\times4$ superlattices but not for the $8\times8$ and $14\times14$ superlattices. The complicated dispersions of the superlattices, particularly for the $8\times8$ and $14\times14$ cases, where there is a significant amount of branch folding, are not Debye-like at the intermediate frequencies, leading to a deviation from the $\omega^{-4}$ scaling. The lifetimes of low-frequency modes for all superlattices are not affected by the mixing and follow a similar $\omega^{-2}$ scaling as seen in the perfect superlattices.
%Because the $\tau\kv$ defined by the spectral width in Eq.(\ref{EQ:lifetime}) corresponds to all possibles mechanisms of phonon interaction, the $\omega^{-4}$ scaling diverges at low frequencies and is replaced by a $\omega^{-2}$ scaling. 

Mixing broadens the power spectra (Fig.~\ref{fig:sed}) and shifts the phonon lifetimes downward (Fig.~\ref{FIG:lifetime}), particularly at the intermediate and higher frequency modes. For the $2 \times 2$ and $ 4 \times 4$ superlattices,  the lifetimes of some higher frequency modes fall below the Ioffe-Regel limit, $\tau_\mathrm{IR} =2\pi/\omega$, where a mode has a lifetime equal to its period of oscillation. The normal modes for a perfect superlattice have a plane wave structure. Under the assumption that the normal modes of a perfect superlattice are representative of a mixed superlattice, reaching the Ioffe-Regel limit is therefore not an indication of spatial localization but rather of temporal localization [see Fig.~\ref{fig:dispersion}(e)]. Modes that are below this limit can thus be considered to be non-propagating delocalized modes (i.e., diffusons) \cite{allen_thermal_1993,allen1999diffusons}. Similar trends in the variation of lifetimes with frequency, dropping below the Ioffe-Regel limit at intermediate frequencies and then rising above at higher frequencies, have also been predicted for LJ alloys \cite{jason2013vc}. Spatial localization has previously been invoked to explain the period-length dependence of superlattice thermal conductivity \cite{PhysRevB.61.3091}, but this mechanism is not applicable to the systems studied here.

%%%
\renewcommand{\textfraction}{0.0}
\begin{figure}%[H]
\begin{center}
%\scalebox{1}{ \includegraphics{/Users/mullspace/Dropbox/git/plots.nogit/images/lifvomega.eps}}
\scalebox{1}{ \includegraphics{lifvomega.eps}}
\renewcommand{\figure}{Fig.}
\caption{Lifetimes for perfect (blue) and 80/20 (red) superlattices. The black line corresponds to the Ioffe-Regel criterion, $2\pi\omega^{-1}$. The brown line corresponds to $\omega^{-2}$ scaling. The green line corresponds to $\omega^{-4}$ scaling. Vertical grey lines correspond to the maximum frequency observed in the bulk systems: for the lighter species, $\omega_{max}=24.7$ and for the heavier species, $\omega_{max}=14.3$. } 
\label{FIG:lifetime}
\end{center}
\end{figure}
%%%

%Fig.~\ref{FIG:lifetime} shows general agreement between the lifetimes calculated using Eq.~(\ref{EQ:tau_eff}) and lifetimes obtained from NMD under the assumption that the eigenvectors of the unmixed systems are valid for the mixed systems; both resolve the $\omega^{-2}$ scaling at low frequencies and the general smearing at intermediate and larger frequencies. At larger frequencies, Tamura theory predicts lifetimes below the Ioffe-Regel limit for all superlattice period lengths.
%%%
\begin{comment}
\begin{table}
\begin{center}
\begin{tabular*}{\textwidth}{c@{\extracolsep{\fill}}ccccc}
\hline\hline\noalign{\smallskip}
&\multicolumn{3}{c}{$N\times N$ Superlattice} \\
\cline{2-5}\noalign{\smallskip}
\hspace{1cm} & $2\times2$ & $4\times4$ & $8\times8$ & $14\times14$  \\
\noalign{\smallskip}\hline\noalign{\smallskip}
%$\tau_{eff}$   & 0.51 $\pm$ 0.51 & 0.49 $\pm$ 0.59 &  0.33 $\pm$ 0.34& 0.21 $\pm$ 0.22 \\
%ALD   & 0 $\pm$ 0 & 0.11  $\pm$  0.14 &  0.09  $\pm$  0.07 & NA \\
RMSE $\sqrt{\frac{\sum_{\pmb{\kappa}\nu}(\tau_{eff}\kv-\tau_{mixed}\kv)^2}{n}}$ &0.65 & 1.21 & 2.09 & 5.42\\
Mean $\overline{\tau_{mixed}\kv}$ &1.37 & 1.93 & 2.74 & 3.40\\
Standard Deviation $\sigma[\tau_{mixed}\kv]$ &2.34 & 2.66 & 5.53 & 11.84\\
\noalign{\smallskip}\hline\hline
\end{tabular*}
\end{center}
\renewcommand{\table}{Table.}
\caption{The root-mean-square error between Tamura theory effective lifetimes and NMD mixed lifetimes. The mean and standard deviation of the NMD mixed lifetimes are provided for context. %ALD was not used for the 14x14 case due to the computational complexity of $\Omega (N_{uc}^4)$.}
}
\label{TB:taud}
\end{table}
\end{comment}
%%%


\subsection{Thermal Conductivity}
\subsection{Size Effects}
Due to their large unit cells, mode-by-mode studies of superlattices are computationally expensive and accounting for size effects in thermal conductivity prediction is challenging. For example, Broido \textit{et al}. were limited to a range of short period lengths ($8\times 8$) using an iterative solution to the BTE for Si/Ge \cite {PhysRevB.70.081310} and the phonon properties for short period Si/Ge superlattices obtained from DFPT were presumed to hold for larger period superlattices \cite{Luckyanova16112012, doi:10.1021/nl202186y}. Likewise, the phonon lifetimes of low frequency modes were extrapolated to a power law from data obtained from a Monte Carlo integration of the BTE \cite{savic:073113}. 

A comparison of the thermal conductivity predictions is presented in Appendix~\ref{appendixc}. The NMD predictions for both the cross-plane and in-plane thermal conductivities varied by 10\% when increasing $N_x$ from six to eight along the cross-plane and fixing $N_y$ and $N_z$ at six ($N_x$ was set to 10 for the $2\times2$ superlattices to resolve the lifetime scalings). %Unlike the previous mode-by mode studies, the phonon properties of short period superlattices was not extended to large period superlattices. 
Due to the scaling of the NMD algorithm, further increasing the Brouillin zone resolution ($N_i$) was computationally prohibitive. As a result of these size effects, the reported thermal conductivity predictions from NMD are presumed to carry an uncertainty of 10\%. Uncertainty in the GK prediction was specified by systematically removing one seed before calculating the thermal conductivity.

We note that one approach to estimating size effects in NMD thermal conductivity predicitions is to conduct simulations for a range of system sizes, plot the inverse of thermal conductivity versus the inverse length of the system, fit a line through the data and then take the vertical axes intercept as the bulk value \cite{PhysRevB.81.214305}.% where the linear behavior of the dispersion near the gamma point of the Brouillin zone and the $\omega^{-2}$ scaling of the lifetimes allowed for the contribution of these unresolved long-wavelength modes to be estimated. \cite{PhysRevB.81.214305} 
This method was not used in previous superlattice studies \cite{doi:10.1021/nl202186y,savic:073113,Luckyanova16112012} and is not used here. The complicated dispersion [Fig.~\ref{fig:dispersion}(a-c)] does not guarantee that this approach is viable and, as such, understanding size effects in superlattices warrants further work.

\begin{table*}
\begin{center}
\begin{tabular*}{\textwidth}{c@{\extracolsep{\fill}}cccccc}
\hline\hline\noalign{\smallskip}
\multicolumn{2}{c}{\multirow{2}{*}{Cross-Plane}}& \multicolumn{4}{c}{$N\times N$ Superlattice} \\
\cline{3-6}\noalign{\smallskip}
\hspace{1cm} && $2\times2$ & $4\times4$ & $8\times8$ & $14\times14$  \\
\noalign{\smallskip}\hline\noalign{\smallskip}
\multirow{3}{*}{Perfect} &NMD & 0.29 $\pm$ 0.03 & 0.23 $\pm$ 0.02 & 0.32 $\pm$ 0.03 & 0.39 $\pm$ 0.04 \\
&GK & 0.25 $\pm$ 0.02 & 0.22 $\pm$ 0.02  &  0.29 $\pm$ 0.02  &  0.39 $\pm$ 0.03\\
&ALD & 0.25 &	0.26  &	0.32	 &0.44\\
&Thermal Circuit & 0.07  &  0.13  &  0.23  &  0.32\\
\noalign{\smallskip}\hline
\multirow{3}{*}{Mixed 80/20} &NMD &0.19 $\pm$ 0.02& 0.17 $\pm$ 0.02& 0.28 $\pm$ 0.03 & 0.42 $\pm$ 0.04\\
&GK  & 0.16 $\pm$ 0.01  &  0.18 $\pm$ 0.02 &  0.29 $\pm$ 0.02 &   0.45 $\pm$ 0.06\\
&Tamura (NMD) & 0.21 $\pm$ 0.02& 0.15 $\pm$ 0.02& 0.30 $\pm$ 0.03& 0.37 $\pm$ 0.04\\
&Tamura (ALD) & 0.12 & 0.19 & 0.26 & 0.40\\
\hline\hline
\end{tabular*}
\end{center}
\renewcommand{\table}{Table.}
\caption{Cross-plane thermal conductivity predictions [W/m-K].}
\label{TB:K_CP}
\end{table*}

\begin{table*}
\begin{center}
\begin{tabular*}{\textwidth}{c@{\extracolsep{\fill}}cccccc}
\hline\hline\noalign{\smallskip}
\multicolumn{2}{c}{\multirow{2}{*}{In-Plane}}&\multicolumn{4}{c}{$N\times N$ Superlattice} \\
\cline{3-6}\noalign{\smallskip}
\hspace{1cm} && $2\times2$ & $4\times4$ & $8\times8$ & $14\times14$  \\
\noalign{\smallskip}\hline\noalign{\smallskip}
\multirow{2}{*}{Perfect} &NMD &0.52 $\pm$ 0.05 & 0.51 $\pm$ 0.05& 0.56 $\pm$ 0.05& 0.60 $\pm$ 0.06\\
&GK &0.53 $\pm$ 0.03 &  0.54 $\pm$ 0.03 &  0.61 $\pm$ 0.05  &  0.66 $\pm$ 0.07 \\
&ALD & 0.55	& 0.53	&	0.59 	&0.63\\
\noalign{\smallskip}\hline
\multirow{3}{*}{Mixed 80/20} & NMD &0.21 $\pm$ 0.02 & 0.25 $\pm$ 0.03 & 0.37 $\pm$ 0.04 & 0.47  $\pm$ 0.05\\
&GK & 0.19 $\pm$ 0.02 &  0.30 $\pm$ 0.01  & 0.43 $\pm$ 0.03 &  0.62 $\pm$ 0.07 \\   
&Tamura (NMD)& 0.22 $\pm$ 0.02 & 0.27 $\pm$ 0.03 & 0.38 $\pm$ 0.04 & 0.45 $\pm$ 0.05\\
&Tamura (ALD) & 0.17 & 0.25 & 0.34 &0.38\\
\hline\hline
\end{tabular*}
\end{center}
\renewcommand{\table}{Table.}
\caption{In-plane thermal conductivity predictions [W/m-K].}
\label{TB:K_IP}
\end{table*}
%%%
\subsubsection{Perfect superlattices}
With reference to Tables~\ref{TB:K_CP} and~\ref{TB:K_IP}, the trends and magnitudes in cross-plane and in-plane thermal conductivity predictions from NMD and GK for perfect superlattices are in good agreement. The in-plane thermal conductivity increases with increasing period length, while the cross-plane thermal conductivity first decreases then increases with period length consistent with previous MD superlattice studies \cite {PhysRevB.77.184302,PhysRevB.72.174302}. The in-plane thermal conductivity is nearly a factor of two larger than cross-plane thermal conductivity for all superlattices because of the larger group velocities which span the entire frequency spectrum for the in-plane direction [see Figs.~\ref{fig:dispersion}(a) and~\ref{fig:dispersion}(c)]. 

A thermal circuit model prediction for cross-plane thermal conductivity using the relation
%%%
\begin{equation}\label{EQ:TCircuit}
\begin{split}
k_{circuit}= \frac{La}{R_{int}+\frac{La}{2k_m}+\frac{La}{2k_{3m}}},
\end{split}
\end{equation}
%%%
is presented in Table~\ref{TB:K_CP}. The boundary resistance for a perfect interface, estimated from the non-equilibrium direct method ($R_{int}=1.4\times10^{-8}$ Km$^2$W$^{-1}$), was combined with bulk thermal conductivities, obtained from NMD, of the lighter material ($k_{m}$=1.2 W/m-K) and the heavier material ($k_{3m}$=0.7 W/m-K) through their respective layer thickness, yielding an effective resistance that was inverted to obtain an effective thermal conductivity. The thermal circuit model underestimates the thermal conductivity for all superlattices, indicating that the bulk phonons of the constituent species are not an accurate description of superlattice phonons. The relative difference decreases with increasing period length, suggesting that this model may become representative of the nature of thermal transport at large enough period lengths.
%%%
%No minimum in cross-plane thermal conductivity is found in the ALD predictions.
\subsubsection{Mixed superlattices}
To assess the assumption of using the perfect superlattice eigenvectors for the mixed superlattices, we use Tamura elastic mass-defect scattering theory to predict their thermal conductivities \cite{tamura_isotope_1983}. In this approach,the lifetimes predicted for perfect superlattices, $\tau_{perfect}\kv$, are modified using the Matthiesen rule %as done by Luckyanova et al. \cite{Luckyanova16112012} 
%%%
\begin{equation}\label{EQ:tau_eff}
\begin{split}
\frac{1}{\tau_{effective}\kv} = \frac{1}{\tau_{defect}\kv}+\frac{1}{\tau_{perfect}\kv} ,
\end{split}
\end{equation}
%%%
where
%%%
\begin{equation}\label{EQ:tau_d}
\begin{split}
\frac{1}{\tau_{defect}\kv} = &\frac{\pi}{2N}\omega^2\kv \sum_{\pmb{\kappa'},\nu'} \delta\left[ \omega\kv - \omega\kvp \right]
\sum_{b} g_2(b) |e^*\kvbap \cdot e\kvba |^2 .
\end{split}
\end{equation}
%%%
Here, $g_2(b)$ is the coupling term for atom $b$ in the unit cell that defines the strength of the mass disordering
%%%
\begin{equation}\label{EQ:g(b)}
\begin{split}
g_2(b) = \sum_\mu c_{\mu}(b)\left[1-\frac{m_{\mu}(b)}{\overline{m(b)}}\right]^2, 
\end{split}
\end{equation}
%%%
where the summation is over the possible species at that atomic position in the unit cell with concentration $c_\mu(b)$, mass $m_\mu(b)$, and average mass $\overline{m(b)}$. Given that there are two atom types in the superlattice unit cell, the lighter atom can be considered to be a mass defect of the heavier portion of the superlattice, and vice-versa. $g_2(b)$ is zero if atom $b$ is unmixed (i.e., for atoms that do not reside within one monolayer of the interface). The delta function in Eq.~(\ref{EQ:tau_d}) is broadened into a Lorentzian function with width on the order of the frequency level spacing (i.e., $\omega \kv -\omega \kvp \approx 0.1$) imposed by the finite size of the systems \cite{allen_thermal_1993}.

For the $2\times 2$ and $4\times 4$ superlattices, the in-plane and cross-plane thermal conductivity predictions are reduced from their corresponding perfect system and approach the 50/50 alloy limit (0.21 W/mK from GK using $N_{x,y,z}=6$). The mixed $2\times 2$ superlattice loses much of its anisotropy between the in-plane and cross-plane directions as there is mixing in all the atomic layers. From Table~\ref{TB:K_CP}, the predictions for mixed superlattices from NMD (using perfect eigenvectors), Tamura theory, and GK follow similar trends, with increasing cross-plane thermal conductivity with increasing period length. From Table~\ref{TB:K_IP}, for mixed superlattices with the exception of $2 \times 2$, NMD and Tamura theory, while in good agreement with each other, predict a lower in-plane thermal conductivity than GK. The discrepancy is more pronounced for the in-plane estimate because the intermediate-frequency modes, which have a non-zero component of group velocity [Fig.~\ref{fig:dispersion}(c)], experience the largest disruption (see Figs.~\ref{fig:sed} and~\ref{FIG:lifetime}). The cross-plane direction, on the other hand, has a near-zero contribution to thermal conductivity from intermediate- and high-frequency modes [Fig.~\ref{fig:dispersion}(a)]. This would also explain the increasing difference between NMD and Tamura theory with GK with increasing with period length, since the number of branches with non-zero components of group velocity increases. Since the GK method does not distinguish between superlattice phonons and disrupted phonons, the GK estimate of thermal conductivity is considered here to be the best result. The discrepancy is a consequence of the disruption by interfacial mixing, through the breakdown of the validity of perfect eigenvectors for NMD and Tamura theory, of the superlattice phonon modes which contribute to thermal transport. The interpretation of superlattice phonons is further discussed in the following section.

%The in-plane components of the group velocities of the disrupted modes may be different than for the perfect superlattice phonons. 
%With increasing period length, these modified modes are more likely to be represented by bulk-like modes and may explain the discrepancy between NMD or Tamura theory and GK.
%The successful application of Tamura's elastic defect scattering theory to systems studied here is consistent with Pomeranchuk's argument that inelastic defect scattering has negligible effects upon thermal transport.\cite{pomeranchuk1942thermal} 
%%%
\begin{comment}
\begin{table}
\begin{center}
\begin{tabular}{lcc}
\hline\hline\noalign{\smallskip}
&\multicolumn{2}{c}{Method} \\
\cline{2-3}\noalign{\smallskip}
$k$ & NMD  & GK  \\
\noalign{\smallskip}\hline\noalign{\smallskip}
Cross-Plane Perfect  & 0.24 $\pm$ 0.02 & 0.22 $\pm$ 0.04\\
Cross-Plane 80/20    & 0.17  $\pm$ 0.01   &   0.18 $\pm$ 0.02 \\
Cross-Plane 60/40    & 0.18  $\pm$ 0.01   &   0.19 $\pm$ 0.02 \\
In-Plane Perfect   & 0.52 $\pm$ 0.03 & 0.54 $\pm$ 0.03  \\
In-Plane 80/20  & 0.25 $\pm$ 0.02 & 0.30 $\pm$ 0.01  \\
In-Plane 60/40   & 0.20 $\pm$ 0.02 & 0.26 $\pm$ 0.01  \\
\noalign{\smallskip}\hline\hline
\end{tabular}
\end{center}
\renewcommand{\table}{Table.}
\caption{A comparison of the thermal conductivity predictions [$Wm^{-1}K^{-1}$] predictions for a $4\times4$ superlattice.}
\label{TB:validate}
\end{table}
\end{comment}
%%%
\begin{comment}
\subsubsection{MFP spectrum}
The plots of the CP thermal conductivity MFP contribution curves, defined by
%%%
\begin{equation}\label{EQ:MFP_contr}
\begin{split}
dk(\lambda)=\lambda\kv d\lambda, 
\end{split}
\end{equation}
%%%
is shown in Fig.~\ref{FIG:MFP_cp}. There is a clear reduction in the contribution to cross-plane thermal conductivity in perfect superlattices from modes with a MFP greater than the period length as period length increases. This trend is consistent with the theoretical predictions from Mahan that a minimum thermal conductivity occurs as the transport behaviour shifts from a wave-regime to particle-regime.\cite{PhysRevLett.84.927,PhysRevB.56.10754} This same trend is observed in the contribution curves for the in-plane conductivity (not shown). The minimum cross-plane thermal conductivity in the for perfect superlattices occurs at a point where the average MFP is slightly greater than the period length ($4 \times 4$). The dimensional value of the peak MFP in perfect superlattices, for both in-plane and cross-plane, does not vary significantly as a function of period length. 

For short period superlattices ($2\times2$, $4\times4$ and $8\times8$), interfacial mixing shifts the average MFP towards the period length and reduces its respective contribution, indicative of the transition of thermal transport from superlattice phonons to thermal transport from phonons which may or may not map to those belonging to the perfect dispersion. This trend further supports the fact that superlattice phonons become disrupted by interfacial mixing, thereby diminishing the enhancing effects of the secondary periodicity upon cross-plane thermal conductivity. At large enough period lengths, the effect of interspecies mixing has a negligible effect on phonon MFP, as the corresponding lifetimes are less affected with increasing period length (Fig.~\ref{FIG:lifetime}) because of the decrease in the number of interfaces per unit of volume. \cite{PhysRevB.79.075316}
%%%
\begin{figure}%[H]
\begin{center}
\scalebox{1}{ \includegraphics{MFP_cp.eps}}
\renewcommand{\figure}{Fig.}
\caption{Phonon mean free path normalized by the period length contribution to the cross-plane thermal conductivity. From top to bottom $2\times2$, $4\times4$, $8\times8$ and $14\times14$ superlattices. Color corresponds to those used in Fig.~\ref{FIG:lifetime}. Average MFP is reported; subscripts correspond to the color of marker [b:blue (perfect), r:red (mixed)]. Orange corresponds to lighter bulk and green corresponds to heavier bulk.}
\label{FIG:MFP_cp}
\end{center}
\end{figure}
%%%

The noise in the contribution distribution at longer MFPs is consequence of the limited resolution of the Brouillin zone enforced by the MD domain. This has been observed in other mode by mode analysis techniques, such as the real space force constant extraction from DFT method used by Esfarjani, where the limited resolution manifested in a stepwise behavior of thermal conductivity accumulation function of bulk silicon. \cite{PhysRevB.84.085204} The linear behavior of the dispersion near the gamma point of the Brouillin zone and the $\omega^{-2}$ scaling of the lifetimes allowed for the contribution of these unresolved long-wavelength modes to be estimated. The linear extrapolation procedure used to predict bulk thermal conductivities \cite{PhysRevB.81.214305} was not used in previous superlattice studies \cite{doi:10.1021/nl202186y,Luckyanova16112012} and is not used here. The complicated dispersion [Fig.~\ref{fig:dispersion}(a-c)] does not guarantee that such an approach is viable, as such, understanding size effects in superlattices warrants further work.
\end{comment}

\section{Superlattice phonons}\label{SEC:sl_phon}

The term ``coherence" has been used in two contexts in relation to phonons. First, coherence is used to describe the modes that emerge from the secondary periodicity (superlattices or a silicon thin films with periodic arrangement of holes \cite{doi:10.1021/nl102918q,PhysRevB.87.195301}). The second context is used to describe the excitation of long-wavelength phonons, usually by femtosecond time-resolved pump-probe techniques \cite{PhysRevLett.73.740,PhysRevB.75.195309}, that do not carry significant thermal energy and are not found in the MD simulations studied here. While both contexts imply the wave picture of vibrational modes, the former context and its relation to thermal transport is the focus of this section.

In a crystalline solid, the phonons that carry thermal energy belong to the dispersion relation of that material, which depends on the geometry, the harmonic force constants, and the constituent masses. The introduction of a secondary periodicity modifies the dispersion such that the modes that emerge do not exist in the composing bulk materials. These non-bulk phonons (in our case, superlattice phonons) propagate and scatter in the periodic structure in a similar manner to what a bulk phonon undergoes in a bulk material. By incorporating mixing at the interfaces, we observed the effect of the disruption of the secondary periodicity on the propagation of superlattice modes. %Coherent effects, for that matter, are found in bulk systems since coherence requires the constructive or destructive interference of waves.

%of vibrational modes when a secondary periodicity is added to a system ( superlattices or a silicon thing file with periodic arrangement of holes),
%The reasons are twofold for the ambiguity attached to the terms \textit {coherent phonon}: its association with exotic configurations of material, like porous silicon \cite{doi:10.1021/nl102918q} where the secondary periodicity emerges from the introduction of the repeated holes or between nanoparticles in a nanofluidic system \cite{Keblinski2002855} and from previous femtosecond time-resolved pump-probe experimental coherent phonon studies \cite{PhysRevLett.73.740,PhysRevB.75.195309} that do not focus upon thermal transport. Coherent effects, for that matter, are found in bulk systems since coherence requires the constructive or destructive interference of waves.

Past literature has cited the superlattice as a structure where coherent effects on thermal transport are possible. The trend in cross-plane thermal conductivity as a function of period length justifies such a claim \cite{PhysRevB.67.195311,PhysRevB.72.174302,PhysRevB.61.3091}, where it is theorized that the minimum thermal conductivity corresponds to the transition from wave-governed transport to particle-governed transport \cite{PhysRevLett.84.927,PhysRevB.56.10754}. %Here, we have observed a minimum in perfect superlattices simply by adopting the correct phonon dispersion for our analysis. 
By using MD simulations, we do not impose any restrictions on the phonon dynamics but let the system move through phase space naturally and thus should sufficiently capture all classical effects, including coherence. The results presented in this work suggest that using the correct normal modes is sufficient to capture the physics of thermal transport in a perfect superlattice, indicating the importance of using the correct dispersion relation. Such dispersion effects are of consequence when one cannot use the bulk material phonon properties to predict thermal transport in a non-bulk-like system. This effect is clearly present for all the systems studied here as evidenced by the discrepancy between the bulk-based thermal circuit model and the NMD and GK predictions (see Table~\ref{TB:K_CP}).

Coherence length, which is defined as the distance a plane wave travels until its phase becomes randomized, only applies to the wave picture of phonon transport. Although there does not appear to be a formal equation for phonon coherence length \cite{chen2005nanoscale}, adopting a unit cell that spans a single superlattice period to generate the modified dispersion relation allows one to ignore the difference between the MFP and coherence length \cite{PhysRevB.67.195311}. This definition is reasonable because the phonons are not localized to a single layer [as confirmed by the participation ratio, see Fig~\ref{fig:dispersion}(e)] and in the perfect superlattice case, the only inelastic scattering mechanism (phase changing) is the anharmonicity of the LJ potential. The plots of the cross-plane and in-plane thermal conductivity accumulation as a function of MFP are shown in Fig.~\ref{FIG:MFP_cuml}. The vertical coordinate of any point on the accumulation function represents the thermal conductivity that comes from phonons with MFP less than the horizontal coordinate of that point.

\begin{comment}
The plots of the CP thermal conductivity MFP contribution curves, defined by
%%%
\begin{equation}\label{EQ:MFP_contr}
\begin{split}
dk_{\alpha}(\Lambda \kv)=c_{ph}\kv v^2_{g,\alpha}\kv \frac{\Lambda \kv} {|\pmb{\mathrm{v}}_{g}\kv|} d\Lambda \kv, 
\end{split}
\end{equation}
%%%
is shown in Fig.~\ref{FIG:MFP_cp}. 
%%%
\begin{figure}%[H]
\begin{center}
\scalebox{1}{ \includegraphics{MFP_cp.eps}}
\renewcommand{\figure}{Fig.}
\caption{Phonon mean free path normalized by the period length contribution to the cross-plane thermal conductivity. Color corresponds to those used in Fig.~\ref{FIG:lifetime}. Average MFP is reported. Orange corresponds to lighter bulk and green corresponds to heavier bulk.}
\label{FIG:MFP_cp}
\end{center}
\end{figure}
%%%
There is a clear reduction in the contribution to cross-plane thermal conductivity in perfect superlattices from modes with a MFP greater than the period length as period length increases. This trend is consistent with the theoretical predictions from Mahan that a minimum thermal conductivity occurs as the transport behaviour shifts from a wave-regime to particle-regime when the average MFP transitions from being greater than the period length to being less than the period length.\cite{PhysRevLett.84.927,PhysRevB.56.10754} This same trend is observed in the contribution curves for the in-plane conductivity (not shown). The minimum cross-plane thermal conductivity for perfect superlattices occurs at a point where the average MFP is greater than the period length ($4 \times 4$), as described by Mahan.\cite{PhysRevLett.84.927} 
\end{comment}

%%%
\begin{figure}%[H]
\begin{center}
\scalebox{1}{ \includegraphics{MFP_cp+ip_cuml_abs.eps}}
\renewcommand{\figure}{Fig.}
\caption{Phonon mean free path accumulation curves.}
\label{FIG:MFP_cuml}
\end{center}
\end{figure}
%%%
In-plane accumulation curves of the perfect and mixed superlattices exhibit asymptotic flattening at longer MFPs. Cross-plane accumulation curves of the perfect and mixed superlattices contain step-like jumps at longer MFPs, a consequence of the the finite resolution of the Brouillin Zone \cite{esfarjani2011heat}. These longer MFPs correspond to the low frequency modes that follow the $\omega^{-2}$ scaling (see Fig.~\ref{FIG:lifetime}). The contributions from the longer MFPs suggest a possible avenue to reducing the cross-plane thermal conductivity through the finite size effect of thin-film superlattices \cite{Luckyanova16112012}.

The major contribution (greater than 90\%) to cross-plane thermal conductivity for all superlattices, perfect and mixed, is attributed to modes with MFPs greater than the superlattice period length. Furthermore, 40\% ($2 \times 2 $) to 60\% ($14 \times 14$) of cross-plane and in-plane conductivity is attributed to modes with MFPs greater than the cross-plane system size ($N_xLa$ for cross-plane and $N_ya$ for in-plane). For the bulk materials, the entire contribution to thermal conductivity is attributed to modes with MFPs greater than the lattice constant with 50\% of the contribution belonging to modes with MFPs greater than the system length. In the perfect superlattice case, the MFPs greater than the period length cannot be interpreted any differently than MFPs greater than the lattice constant of a bulk crystalline structure.

Mixing shifts the in-plane thermal conductivity contribution to shorter MFPs for the $2 \times 2 $ and $4 \times 4 $ superlattices. Differences between perfect and mixed accumulations curves for cross-plane thermal conductivity manifest at intermediate and longer MFPs. These differences for the in-plane and cross-plane, however, become increasingly small with increasing period length. Furthermore, mixing does not change the range of MFPs. 

We do not find evidence that, for the range of superlattices studied here, that superlattice phonons require a different theoretical treatment than bulk phonons. The concept of coherent effects can be argued to be purely interpretational and is not required to model phonon transport in these superlattices. There remains the question, however, of whether or not this approach will hold for larger superlattice period lengths.

% MFPs greater than the superlattice period length do not manifest because of phase-preserving reflection from the interfaces.

%This statement is confirmed by observing that MFP contribution distribution, normalized by the superlattice period length, for the lighter (orange) and heavier (green) bulk systems (Fig.~\ref{FIG:MFP_cp}) follows the MFP distribution for the perfect superlattices. We do not find evidence that superlattice phonons experience coherent effects and believe that thermal transport in perfect superlattices is accurately represented by the corresponding dispersion relation.%The dimensional value of the peak MFP in perfect superlattices, for both in-plane and cross-plane, does not vary significantly as a function of period length.

%For short period superlattices ($2\times2$, $4\times4$ and $8\times8$), interfacial mixing shifts the average MFP towards the period length and reduces its respective contribution, indicative of the transition of thermal transport from superlattice phonons to thermal transport from phonons which may or may not map to those belonging to the perfect dispersion. %This trend further supports the fact that superlattice phonons become disrupted by interfacial mixing, thereby diminishing the enhancing effects of the secondary periodicity upon cross-plane thermal conductivity. 
%The assumption of the validity of perfect superlattice eigenvectors captures the downward shift in the MFP distributions and generates a discrepancy between GK and Tamura theory or NMD predictions for in-plane thermal conductivity. These results suggest that roughness and disorder disrupt the dispersion effect. At large enough period lengths ($14\times14$), the effect of interspecies mixing has a negligible effect on phonon MFP, as the corresponding lifetimes are less affected with increasing period length (Fig.~\ref{FIG:lifetime}) because of the decrease in the number of interfaces per unit of volume. \cite{PhysRevB.79.075316}

%Mixing introduces an additional scattering mechanism, modeled in Tamura theory as an elastic mass defect. Elastic scattering implies that the wavevector of the incident phonon does not undergo a change in magnitude but only experiences a change in direction and thus is a phase preserving event. However, the use of the Matthiesen rule effectively masks any distinction between anharmonic scattering and defect scattering by encapsulating the overall effect as inelastic scattering. The assumption of the validity of perfect superlattice eigenvectors captures the downward shift in the MFP distributions but generates a discrepancy between GK and Tamura theory or NMD predictions for in-plane thermal conductivity. These results suggest that roughness and disorder disrupt the dispersion effect.% and non-superlattice phonons, but not necessarily bulk-like phonons, may be more representative.%, consequently destroying coherent effects (inelastic scattering may be more representative).\cite{PhysRevB.67.195311,dames_682} 

%Phase is assumed to be randomized through inelastic phonon-phonon, but not elastic boundary or impurity scattering \cite{chen2005nanoscale}. Furthermore, the inability for NMD and ALD to estimate the in-plane thermal conductivity in mixed superlattices indicates the disruption of the superlattice dispersion. 

%Finally, by relying upon the BTE, any coherent effect is abstracted into the phonon-as-a-particle interpretation.
\section{Summary}

Phonon properties in perfect and mixed LJ superlattices where predicted using NMD. Differences between in-plane and cross-plane components of group velocity are responsible for the respective differences between thermal conductivity [Fig.~\ref{fig:dispersion}(a)-(c)]. We find $\omega^{-2}$ lifetime scaling at low-frequency modes in perfect and mixed superlattices and $\omega^{-4}$  lifetime scaling at intermediate frequencies in mixed superlattices (Fig.~\ref{FIG:lifetime}). Interspecies mixing disrupts the secondary periodicity [Fig.~\ref{fig:sed}(B)] and reduces phonon lifetimes, thereby reducing in-plane and cross-plane thermal conductivity (Table~\ref{TB:K_CP} and Table~\ref{TB:K_IP}). Discrepancies between the in-plane thermal conductivity predictions for mixed superlattices from GK with NMD and Tamura theory are attributed to the use of perfect eigenvectors. We note that the cross-plane thermal conductivity can be reduced by truncating the contribution from longer MFPs through the finite size effect in thin film superlattices [Fig.~\ref{FIG:MFP_cuml}].


%% This adds a line for the Bibliography in the Table of Contents.
\addcontentsline{toc}{chapter}{Bibliography}
%% *** Set the bibliography style. ***
%% (change according to your preference/requirements)
\bibliographystyle{plain}
%% *** Set the bibliography file. ***
%% ("thesis.bib" by default; change as needed)
\bibliography{thesis}

\appendix
\chapter{The importance of eigenvectors}

The difference between Thomas and Larkin's formulation of the SED can be understood through the properties of the eigenvectors. By expanding the kinetic term of the Hamiltonian
\begin{equation}
\begin{split}
\dot{Q}(\bm{\kappa},\nu)\dot{Q}^*(\bm{\kappa},\nu)=\frac{1}{N}[\sqrt{m_1}exp(-i\bm{\kappa}\cdot\bm{r}(1))\bm{e}^*(1,\bm{\kappa},\nu)\cdot\dot{\bm{u}}(1,t)\\
+\sqrt{m_2}exp(-i\bm{\kappa}\cdot\bm{r}(2))\bm{e}^*(2,\bm{\kappa},\nu)\cdot\dot{\bm{u}}(2,t)\\
+...\sqrt{m_n}exp(-i\bm{\kappa}\cdot\bm{r}(n))\bm{e}^*(n,\bm{\kappa},\nu)\cdot\dot{\bm{u}}(n,t)\\
+...\sqrt{m_N}exp(-i\bm{\kappa}\cdot\bm{r}(N))\bm{e}^*(N,\bm{\kappa},\nu)\cdot\dot{\bm{u}}(N,t)]\\
\times[\sqrt{m_1}exp(i\bm{\kappa}\cdot\bm{r}(1))\bm{e}(1,\bm{\kappa},\nu)\cdot\dot{\bm{u}}(1,t)\\
+\sqrt{m_2}exp(i\bm{\kappa}\cdot\bm{r}(2))\bm{e}(2,\bm{\kappa},\nu)\cdot\dot{\bm{u}}(2,t)\\
+...\sqrt{m_n}exp(i\bm{\kappa}\cdot\bm{r}(n))\bm{e}(n,\bm{\kappa},\nu)\cdot\dot{\bm{u}}(n,t)\\
+...\sqrt{m_N}exp(i\bm{\kappa}\cdot\bm{r}(N))\bm{e}(N,\bm{\kappa},\nu)\cdot\dot{\bm{u}}(N,t)]
\end{split}
\end{equation}
From solving the eigenvalue problem of lattice dynamics, the eigenvectors take the form
\begin{equation}
\bm{e}(\bm{\kappa},\nu)=
\begin{pmatrix}
\bm{e}(1,\bm{\kappa},\nu)\\
\bm{e}(2,\bm{\kappa},\nu)\\
...\\
\bm{e}(n,\bm{\kappa},\nu)\\
\end{pmatrix}
\end{equation}
where $n$ is the number of atoms in the unit cell and $N$ is the total number of atoms
\begin{equation}
\bm{e}(1,\bm{\kappa},\nu)=\bm{e}(n+1,\bm{\kappa},\nu)
\end{equation}
\begin{equation}
\bm{e}(n,\bm{\kappa},\nu)=\bm{e}(N,\bm{\kappa},\nu)
\end{equation}
Recalling that the orthogonality of the eigenvectors ensures 
\begin{equation}
\sum_{j}\bm{e}(j,\bm{\kappa},\nu)\cdot\bm{e}^*(j,\bm{\kappa},\nu)= \delta_{\bm{\kappa},\nu:\bm{\kappa},\nu'}
\end{equation}
\begin{equation}
\bm{e}(\bm{\kappa},\nu)\cdot\bm{e}^*(\bm{\kappa},\nu)= \delta_{\bm{\kappa},\nu:\bm{\kappa},\nu'}
\end{equation}
For the sake of argument, assume this implies
\begin{equation}
\sum_{n'}\bm{e}(n,\bm{\kappa},\nu)\cdot\bm{e}^*(n',\bm{\kappa},\nu)=\delta_{n:n'}
\end{equation}
If so
\begin{equation}
\dot{Q}(\bm{\kappa},\nu)\dot{Q}^*(\bm{\kappa},\nu)=|\frac{1}{\sqrt{N}}\sum_{jl}\sqrt{m_j}\dot{\bm{u}}(jl,t)|^2
\end{equation}
which is the average kinetic energy of an atom. In theory, the orthogonality applies to the entire eigenvector $\bm{e}(\bm{\kappa},\nu)$ but does not imply orthogonality between its components
\begin{equation}
\sum_{n'}\bm{e}(n,\bm{\kappa},\nu)\cdot\bm{e}^*(n',\bm{\kappa},\nu)\neq\delta_{n:n'}
\end{equation}
It is therefore necessary to project the velocities onto the eigenvectors before calculating the autocorrelation and the SED, since $\dot{Q}(\bm{\kappa},\nu)\dot{Q}^*(\bm{\kappa},\nu)$ will have some form resembling the initial expansion in Equation 40.



\chapter{Insight into the presence of additional peaks}\label{appendix:b}
Consider a linear, nearest neighbour, monotomic chain a with periodic boundary condition, the equations of motions are:
\begin{equation}
\begin{split}
m\ddot{x}_1&=k(x_2-x_1)+k(x_n-x_1)\\
m\ddot{x}_2&=k(x_3-x_2)+k(x_4-x_2)\\
...\\
m\ddot{x}_n&=k(x_{N-1}-x_N)+k(x_1-x_N)\\
\end{split}
\end{equation}
which is a set of coupled linear differential equations. Removing a single equation and the system becomes underdetermined (infinite number of solutions). The matrix version of this system is:
\begin{equation}
\bm{M}\bm{\ddot{X}}-\bm{K}\bm{X}=0
\end{equation}
With the usual harmonic assumption $\bm{X}=\bm{E}e^{i\omega t}$, the problem is recast:
\begin{equation}
[\omega^2\bm{M}^{-1}\bm{M}-\bm{M}^{-1}\bm{K}]\bm{E}=0
\end{equation}
Let
\[
\bm{\psi}=
\begin{bmatrix}
   \bm{E}_1 & \bm{E}_2 & \dots &\bm{E}_n \\
 \end{bmatrix}
\]
We can write $\bm{\psi}$ as a transformation matrix
\begin{equation}
\begin{split}
\bm{X}&=\bm{\psi}\bm{Q}\\
\bm{Q}&=\bm{\psi}^{-1}\bm{X}
\end{split}
\end{equation}
In relation to the dividing the domain into subdomains, we want to determine $\bm{Q}$ using a reduced set of $\bm{X}$, given $\bm{\psi}$. Looking at a 4 MDOF case, the explicit matrix representation is:
\[
\begin{bmatrix}
   Q_1\\
   Q_2\\
   Q_3\\
   Q_4\\
\end{bmatrix}=
\begin{bmatrix}
   E_{11} & E_{12} & E_{13} & E_{14} \\
   E_{21} & E_{22} & E_{23} & E_{24} \\
   E_{31} & E_{32} & E_{33} & E_{34} \\
   E_{41} & E_{42} & E_{43} & E_{44} \\
\end{bmatrix}^{-1}
\begin{bmatrix}
   x_1\\
   x_2\\
   x_3\\
   x_4\\
\end{bmatrix}
\]
Given a reduced set of $\bm{X}= [x_1,x_2]$, the equations are rearranged
\begin{equation}
\begin{split}
Q_1-E_{13}'x_3-E_{14}'x_4&= E_{11}'x_1 + E_{12}'x_2\\
Q_2-E_{23}'x_3-E_{24}'x_4&= E_{21}'x_1 + E_{12}'x_2\\
Q_3-E_{33}'x_3-E_{34}'x_4&= E_{31}'x_1 + E_{32}'x_2\\
Q_4-E_{43}'x_3-E_{44}'x_4&= E_{41}'x_1 + E_{42}'x_2.
\end{split}
\end{equation}
This results shows that the transformation from $\bm{X}$ to $\bm{Q}$ is relies upon missing information contained with $x_3$ and $x_4$. The original coupled equations of motions are:
\begin{align}
m\ddot{x}_1&=k(x_2-x_1)+k(x_4-x_1)\\
m\ddot{x}_2&=k(x_3-x_2)+k(x_1-x_2)\\
m\ddot{x}_3&=k(x_4-x_3)+k(x_2-x_3)\\
m\ddot{x}_4&=k(x_1-x_4)+k(x_3-x_4)
\end{align}
The eigenvalue problem takes the form:
\[
\begin{bmatrix}
   m & 0 & 0 & 0 \\
   0 & m & 0 & 0 \\
   0 & 0 & m & 0 \\
   0 & 0 & 0 & m \\
\end{bmatrix}^{-1}
\begin{bmatrix}
   2k & -k & 0 & -k \\
   -k & 2k & -k & 0 \\
   0 & -k & 2k & -k \\
   -k & 0 & -k & 2k \\
\end{bmatrix}
\begin{bmatrix}
   E_{n,1} \\
   E_{n,2} \\
   E_{n,3} \\
   E_{n,4} \\
\end{bmatrix}
=\omega^2_n
\begin{bmatrix}
   E_{n,1} \\
   E_{n,2} \\
   E_{n,3} \\
   E_{n,4} \\
\end{bmatrix}
\]
Letting $m=1$ and $k=1$, the relation between coupled coordinates and normal mode coordinates is, $\pmb{\dot{X}}=\pmb{\psi}\pmb{\dot{Q}}$:
\[
\begin{bmatrix}
   \dot{x}_1\\
   \dot{x}_2\\
   \dot{x}_3\\
   \dot{x}_4\\
\end{bmatrix}=
\begin{bmatrix}
  0.50000 & -0.50000  & 0.70711 & 0.00000 \\
  0.50000 &  0.50000 &  0.00000 & -0.70711 \\ 
  0.50000 & -0.50000 & -0.70711 & 0.00000  \\
  0.50000 &  0.50000 &  0.00000 & 0.70711  \\
\end{bmatrix}
\begin{bmatrix}
   \dot{Q}_1\\
   \dot{Q}_2\\
   \dot{Q}_3\\
   \dot{Q}_4\\
\end{bmatrix}
\]
If we take the 2 atom case:
\[
\begin{bmatrix}
   m & 0  \\
   0 & m  \\
\end{bmatrix}^{-1}
\begin{bmatrix}
    k & -k  \\
   -k & k  \\
\end{bmatrix}
\begin{bmatrix}
   E_{n,1} \\
   E_{n,2} \\
\end{bmatrix}
=\omega^2_n
\begin{bmatrix}
   E_{n,1} \\
   E_{n,2} \\
\end{bmatrix}
\]
\[
\begin{bmatrix}
   \dot{P}_1\\
   \dot{P}_2\\
\end{bmatrix}=
\begin{bmatrix}
  0.70711 &  -0.70711 \\
  0.70711 &  0.70711 \\
\end{bmatrix}^{-1}
\begin{bmatrix}
   \dot{x}_1\\
   \dot{x}_2\\
\end{bmatrix}
\]
Expressing $P$ in terms of $Q$:
\begin{align}
\dot{P}_1=\frac{\sqrt{2}\dot{Q}_1}{2}+\frac{\dot{Q}_3}{2}-\frac{\dot{Q}_4}{2}\\
\dot{P}_2=\frac{\sqrt{2}\dot{Q}_2}{2}-\frac{\dot{Q}_3}{2}-\frac{\dot{Q}_4}{2}\\
%P_1=\frac{\sqrt{2}Q_2}{2}+\frac{\sqrt{2}Q_4}{2}\\
%P_2=-\frac{\sqrt{2}Q_1}{2}-\frac{\sqrt{2}Q_3}{2}
\end{align}
Without taking $<\dot{Q}(\tau)\dot{Q}(0)>$, the Fourier transform of the set of $\dot{Q}$ using the output velocites from MD gives
\begin{figure}[!h]
\centering
\includegraphics[trim = 40mm 0mm 10mm 0mm,scale=0.5,angle=0]{noxcorr.eps}
\caption{Plots of power spectrums.}
\label{fig:awesome_image}
\end{figure}



%% *** NOTE ***
%% If you don't use bibliography files, comment out the previous line
%% and use \begin{thebibliography}...\end{thebibliography}.  (In that
%% case, you should probably put the bibliography in a separate file and
%% `\include' or `\input' it here).

\end{document}
