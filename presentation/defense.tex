\documentclass{beamer}
\usepackage{graphicx}
\usepackage{epstopdf}
\usepackage{multirow}
\usepackage{booktabs}
\usepackage{xcolor}

\usetheme[
        %url={nano.uoft.ca},
        %numbering={false},
        menuwidth={0.3\paperwidth}
        ]{uoft}

\setbeamercovered{transparent=20}
\graphicspath{
{/home/schuberm/Dropbox/git/plots.nogit/images/}
{/Users/mullspace/Dropbox/git/plots.nogit/images/}}

%--------------------------------------------------------------------------
%DEFINE COMMANDS
%--------------------------------------------------------------------------
\newcommand{\EXP}[1]{\exp\mspace{-5.0mu}\left[#1\right]\mspace{-3.0mu}}

\newcommand{\SUM}[2]{\ifthenelse{\equal{#1}{0}}{\sum_{
\alpha_{#2},b_{#2},l_{#2}}^{3,4L,N}} {\ifthenelse{\equal{#1}{1}}{\sum_{
\alpha_{#2},b_{#2}}^{3,n}}{\sum_{\pmb{\kappa}#2,\nu#2}^{N,3n}}}}

\newcommand{\ab}[2]{\mspace{-4.0mu}\left(\mspace{-8.0mu}
\begin{smallmatrix}&\ifthenelse{\equal{#1}{}}{a}{#1} \\&\ifthenelse
{\equal{#2}{}}{b}{#2}\end{smallmatrix}\mspace{-3.0mu}\right)}

\newcommand{\lO}{\mspace{-4.0mu}\left(\mspace{-8.0mu}
\begin{smallmatrix}&l \\&0\end{smallmatrix}
\mspace{-3.0mu}\right)}

\newcommand{\kvba}{\mspace{-4.0mu}\left(\mspace{-8.0mu}
\begin{smallmatrix} &\pmb{\kappa} &b \\ &\nu &\alpha\end{smallmatrix}
\mspace{-3.0mu}\right)}

\newcommand{\kvbap}{\mspace{-4.0mu}\left(\mspace{-8.0mu}
\begin{smallmatrix} &\pmb{\kappa}' &b \\ &\nu' &\alpha\end{smallmatrix}
\mspace{-3.0mu}\right)}

\newcommand{\kvt}{\mspace{-4.0mu}\left(\mspace{-8.0mu}
\begin{smallmatrix}&\pmb{\kappa} \\&\nu\end{smallmatrix}
\mspace{-2.0mu},t\right)}

\newcommand{\kvw}{\mspace{-4.0mu}\left(\mspace{-8.0mu}
\begin{smallmatrix}&\pmb{\kappa} \\&\nu\end{smallmatrix}
\mspace{-2.0mu},\omega\right)}

\newcommand{\kv}{\mspace{-4.0mu}\left(\mspace{-8.0mu}
\begin{smallmatrix}&\pmb{\kappa} \\&\nu\end{smallmatrix}
\mspace{-3.0mu}\right)}

\newcommand{\kvp}{\mspace{-4.0mu}\left(\mspace{-8.0mu}
\begin{smallmatrix}&\pmb{\kappa'} \\&\nu'\end{smallmatrix}
\mspace{-3.0mu}\right)}

\newcommand{\kw}{\mspace{-4.0mu}\left(\mspace{-8.0mu}
\begin{smallmatrix}&\pmb{\kappa} \\&\omega\end{smallmatrix}
\mspace{-3.0mu}\right)}

\newcommand{\lbt}{\mspace{-4.0mu}\left(\mspace{-8.0mu}
\begin{smallmatrix}&l \\&b\end{smallmatrix}\mspace{-2.0mu},t\right)}
%--------------------------------------------------------------------------
%END COMMANDS
%--------------------------------------------------------------------------

\begin{document}


%%%%%%%%%%%%%%%%%%%%%%%%%%%%%%%%%%%%%%%%%%%%%%%%%%%%%%
%%%%%%%%%%%%%%%%%%%%%%%%%%%%%%%%%%%%%%%%%%%%%%%%%%%%%%
\begin{frame}
\Large{Phonon Properties in Superlattices}
%\subtitle{SUBTITLE}

\small{Samuel Huberman}\\

\date{
	\\
	\vspace{1cm}
	\today
}
\end{frame}

%%%%%%%%%%%%%%%%%%%%%%%%%%%%%%%%%%%%%%%%%%%%%%%%%%%%%%
%%%%%%%%%%%%%%%%%%%%%%%%%%%%%%%%%%%%%%%%%%%%%%%%%%%%%%
\section{Background and Motivation}
\begin{frame}{Why Study Nanoscale Energy Transport}
\begin{itemize}
\item Power dissipation problem for CPUs and GPUs
\begin{itemize}
\item Increasing clock speed requires more power
\end{itemize}
\item Improve performance of solar and thermoelectric technologies
\begin{itemize}
\item Figure of merit:
%%%
\begin{equation}\label{EQ:NMD:qdot}
\begin{split}
ZT=\frac{\sigma S^2 T}{k}
\end{split}
\end{equation}
%%%
\end{itemize}
\end{itemize}
\end{frame}

%%%%%%%%%%%%%%%%%%%%%%%%%%%%%%%%%%%%%%%%%%%%%%%%%%%%%%
%%%%%%%%%%%%%%%%%%%%%%%%%%%%%%%%%%%%%%%%%%%%%%%%%%%%%%
\begin{frame}{Funny things at the Nanoscale}
\begin{itemize}
%\item As size of system decreases, transition from diffusive to ballistic transport
\item Fourier's Law fails
%%%
\begin{equation}\label{EQ:NMD:qdot}
\begin{split}
Q\neq-k\frac{dT}{dx}
\end{split}
\end{equation}
%%%

\item Thermal conductivity becomes a function of length, $k(L)$

\item What is responsible for this phenomena?
\end{itemize}
\end{frame}

%%%%%%%%%%%%%%%%%%%%%%%%%%%%%%%%%%%%%%%%%%%%%%%%%%%%%%
%%%%%%%%%%%%%%%%%%%%%%%%%%%%%%%%%%%%%%%%%%%%%%%%%%%%%%
\begin{frame}{How to Study Nanoscale Energy Transport}
Phonons
\begin{itemize}
\item Propagating normal modes of vibration
\item Delocalized in space and time
\item Identified by the mode tuple ($\pmb{\kappa}$, $\nu$)
\item Thermal conductivity in terms of phonon properties:
%%%
\begin{equation}\label{EQ:M:conductivity}
\begin{split}
k_{\mathbf{\alpha}}=&\sum_{\nu,\pmb{\kappa}}^{3n,N} \textcolor{teal}{c_{ph}\kv}
\textcolor{red}{v^{2}_{g,\mathbf{\alpha}}\kv} \textcolor{blue}{\tau\kv}
\end{split}
\end{equation}
%%%
\begin{itemize}
\item $\textcolor{teal}{c_{ph}\kv}$: Mode-dependent volumetric specific heat
\item $\textcolor{red}{\pmb{\mathrm{v}}_{g}\kv}$: Mode group velocity
\item $\textcolor{blue}{\tau\kv}$: Mode lifetime
\end{itemize}

\end{itemize}
\textcolor{orange}{CHALLENGE: PREDICT PHONON PROPERTIES}
\end{frame}

%%%%%%%%%%%%%%%%%%%%%%%%%%%%%%%%%%%%%%%%%%%%%%%%%%%%%%
%%%%%%%%%%%%%%%%%%%%%%%%%%%%%%%%%%%%%%%%%%%%%%%%%%%%%%
\section{Methodology}
\begin{frame}{How to Study Nanoscale Energy Transport: Molecular Dynamics}
Numerically integrate Newton's equation of motion with an empirically defined interaction
\begin{equation}
\begin{split}
\pmb{\mathrm{F}}&=m\pmb{\mathrm{a}}\\
\pmb{\mathrm{F}}&=\frac{dE}{d\pmb{\mathrm{r}}}\\
E(r)&=-4\epsilon\left[\left(\frac{\sigma}{r}\right)^6-\left(\frac{\sigma}{r}\right)^{12}\right]
\end{split}
\end{equation}

Since MD is classical, mode dependent volumetric specific heat is independent of temperature
\begin{equation}\label{EQ:Cph}
\textcolor{teal}{c_{ph}\kv}=\frac{\partial E}{V\partial T}=\frac{k_B}{V}	
\end{equation}
\end{frame}

%%%%%%%%%%%%%%%%%%%%%%%%%%%%%%%%%%%%%%%%%%%%%%%%%%%%%%
%%%%%%%%%%%%%%%%%%%%%%%%%%%%%%%%%%%%%%%%%%%%%%%%%%%%%%
\begin{frame}{How to Study Nanoscale Energy Transport: Lattice Dynamics}

$D(\pmb{\kappa})$ contains information about the masses, the geometry and second-order derivatives of interatomic potential.

Solve the eigenvalue problem to obtain frequencies (eigenvalues) and normal modes (eigenvectors)

\begin{equation}
[D(\pmb{\kappa})-I\omega^2\kv]\pmb{\mathrm{e}}\kv = 0
\end{equation}

Group velocities from finite differencing
\begin{equation}\label{EQ:NMD:vg}
\begin{split}
\textcolor{red}{\pmb{\mathrm{v}}_{g}\kv}=\frac{\partial \omega \kv}{\partial \pmb{\kappa}}
\end{split}
\end{equation}

Determined $\textcolor{teal}{c_{ph}\kv}$, $\textcolor{red}{\pmb{\mathrm{v}}_{g}\kv}$. What about $\textcolor{blue}{\tau\kv}$?

\end{frame}

%%%%%%%%%%%%%%%%%%%%%%%%%%%%%%%%%%%%%%%%%%%%%%%%%%%%%%
%%%%%%%%%%%%%%%%%%%%%%%%%%%%%%%%%%%%%%%%%%%%%%%%%%%%%%
\begin{frame}{Normal Mode Decomposition}
Time derivative of the normal mode coordinate
%%%
\begin{equation}\label{EQ:NMD:qdot}
\begin{split}
\dot{q}\kvt{}{}{}=&\sum_{\alpha, b, l}^{3,4L,N}\sqrt{\frac{m_b}{N}}\textcolor{olive}{\dot{u}_{\alpha}\lbt}\textcolor{purple}{e^*\kvba}\EXP{i\pmb{\kappa}\cdot\mathbf{r}_0\lO}
\end{split}
\end{equation}
%%%
\begin{itemize}
\item $\textcolor{olive}{\dot{u}_{\alpha}\lbt}$: Atomic velocity from MD
\item $\textcolor{purple}{e^*\kvba}$: Eigenvector from LD
\end{itemize}
\end{frame}

%%%%%%%%%%%%%%%%%%%%%%%%%%%%%%%%%%%%%%%%%%%%%%%%%%%%%%
%%%%%%%%%%%%%%%%%%%%%%%%%%%%%%%%%%%%%%%%%%%%%%%%%%%%%%
\begin{frame}{Normal Mode Decomposition}
Power spectrum by Fourier Transform
%%%
\begin{equation}\label{EQ:NMD:SED}
\begin{split}
T\kvw=&\lim_{\tau_0\rightarrow\infty}\frac{1}{2\tau_0}\left|\frac{1}{\sqrt{2\pi}}\int_{0}^{\tau_0}\dot{q}\kvt\exp(-i\omega t)dt\right|^2
\end{split}
\end{equation}
%%%
Expected to have Lorentzian form
%%%
\begin{equation}\label{EQ:NMD:LOR}
T\kvw \approx C_0\kv\frac{\Gamma\kv/\pi}{[\omega_0\kv-\omega]^2+\Gamma^2\kv}
\end{equation}
%%%
Fitting yields the lifetime
%%%
\begin{equation}\label{EQ:lifetime}
\textcolor{blue}{\tau\kv}=\frac{1}{2\Gamma\kv}
\end{equation}
%%%

\end{frame}
%%%%%%%%%%%%%%%%%%%%%%%%%%%%%%%%%%%%%%%%%%%%%%%%%%%%%%
%%%%%%%%%%%%%%%%%%%%%%%%%%%%%%%%%%%%%%%%%%%%%%%%%%%%%%
\begin{frame}{NMD vs ALD for bulk}
\begin{figure}[t]
\begin{center}
\vspace*{-0.8cm}
\scalebox{0.75}{\includegraphics{NMD_v_ALD_bulk.eps}}
\renewcommand{\figure}{Fig.}
\label{fig:nmd_v_ald_bulk}
\end{center}
\end{figure}
\end{frame}

%%%%%%%%%%%%%%%%%%%%%%%%%%%%%%%%%%%%%%%%%%%%%%%%%%%%%%
%%%%%%%%%%%%%%%%%%%%%%%%%%%%%%%%%%%%%%%%%%%%%%%%%%%%%%
\begin{frame}{Superlattices}
Thermoelectric applications

\end{frame}


%%%%%%%%%%%%%%%%%%%%%%%%%%%%%%%%%%%%%%%%%%%%%%%%%%%%%%
%%%%%%%%%%%%%%%%%%%%%%%%%%%%%%%%%%%%%%%%%%%%%%%%%%%%%%
\begin{frame}{\small{4x4 MD Domain}}
\begin{figure}[t]
\begin{center}
\vspace*{-0.8cm}
\scalebox{0.55}{\includegraphics{4p_ai.eps}}
\renewcommand{\figure}{Fig.}
\label{fig:md_domain}
\end{center}
\end{figure}
\end{frame}

%%%%%%%%%%%%%%%%%%%%%%%%%%%%%%%%%%%%%%%%%%%%%%%%%%%%%%
%%%%%%%%%%%%%%%%%%%%%%%%%%%%%%%%%%%%%%%%%%%%%%%%%%%%%%
\section{Results}
\subsection{Dispersion}
\begin{frame}{\small{Dispersion}}
\begin{figure}[!h]
\vspace*{-0.6cm}
\begin{center}
\scalebox{0.48}{ \includegraphics{dispersion.eps}}
\renewcommand{\figure}{Fig.}
\label{fig:sed}
\end{center}
\end{figure}
\begin{figure}[!h]
\vspace*{-0.4cm}
\begin{center}
\scalebox{0.48}{ \includegraphics{dispersion_ip.eps}}
\renewcommand{\figure}{Fig.}
\label{fig:dispersion}
\end{center}
\end{figure}
\end{frame}


%%%%%%%%%%%%%%%%%%%%%%%%%%%%%%%%%%%%%%%%%%%%%%%%%%%%%%
%%%%%%%%%%%%%%%%%%%%%%%%%%%%%%%%%%%%%%%%%%%%%%%%%%%%%%
\subsection{Power Spectra}
\begin{frame}{Power Spectrum}
\begin{columns}
\column{.075\textwidth}
\begin{figure}[t]
\vspace*{-1cm}
\hspace*{-0.9cm}
%\begin{center}
\includegraphics[height=70mm]{4p_dispersion_only.eps}
\renewcommand{\figure}{Fig.}
\label{fig:disp_4p}
%\end{center}
\end{figure}
\column{.91\textwidth}
\begin{figure}[t]
%\begin{center}
\vspace*{-1.2cm}
\hspace*{1.9cm}
\scalebox{0.60}{\includegraphics{sed.eps}}
\renewcommand{\figure}{Fig.}
%\caption{(Colour online) Spectral energy density plots for the selected modes along k=[1,0,0] of a 4x4 superlattice as shown in Figure~\ref{fig:dispersion}. Dark blue corresponds to a superlattice without mixing, red corresponds to mixing of 80/20 and light blue corresponds to mixing of 60/40.}
\label{fig:sed}
%\end{center}
\end{figure}
\end{columns}
\end{frame}

%%%%%%%%%%%%%%%%%%%%%%%%%%%%%%%%%%%%%%%%%%%%%%%%%%%%%%
%%%%%%%%%%%%%%%%%%%%%%%%%%%%%%%%%%%%%%%%%%%%%%%%%%%%%%
\subsection{Lifetimes}
\begin{frame}{Lifetimes}
\begin{figure}[!h]
\begin{center}
\vspace*{-0.8cm}
\scalebox{0.55}{\includegraphics{lifvomega_defense.eps}}
\renewcommand{\figure}{Fig.}
\label{fig:lifetimes}
\end{center}
\end{figure}
\end{frame}

%%%%%%%%%%%%%%%%%%%%%%%%%%%%%%%%%%%%%%%%%%%%%%%%%%%%%%
%%%%%%%%%%%%%%%%%%%%%%%%%%%%%%%%%%%%%%%%%%%%%%%%%%%%%%
\subsection{MFP}
\begin{frame}{MFP}
\begin{figure}[t]
\begin{center}
\vspace*{-0.8cm}
\scalebox{0.55}{\includegraphics{MFP_cp_defense.eps}}
\renewcommand{\figure}{Fig.}
\label{fig:mfp_contribution}
\end{center}
\end{figure}
\end{frame}

%%%%%%%%%%%%%%%%%%%%%%%%%%%%%%%%%%%%%%%%%%%%%%%%%%%%%%
%%%%%%%%%%%%%%%%%%%%%%%%%%%%%%%%%%%%%%%%%%%%%%%%%%%%%%
%\begin{frame}{GK}

%\begin{figure}%[H]
%\begin{center}
%\includegraphics[width=0.3\textwidth]{GK_cp5.eps} \hspace{0.05\textwidth}%
%\includegraphics[width=0.3\textwidth]{/home/schuberm/Dropbox/git/plots.nogit/images/GK_cp7.eps} \\[2em]
%\end{center}
%\end{figure}
%\column{.5\textwidth}
%\begin{figure}%[H]
%\begin{center}
%\includegraphics[width=0.3\textwidth]{/home/schuberm/Dropbox/git/plots.nogit/images/GK_cp8.eps} \hspace{0.05\textwidth}%
%\includegraphics[width=0.3\textwidth]{/home/schuberm/Dropbox/git/plots.nogit/images/GK_cp9.eps} \par
%\end{center}
%\end{figure}

%\end{frame}

%%%%%%%%%%%%%%%%%%%%%%%%%%%%%%%%%%%%%%%%%%%%%%%%%%%%%%
%%%%%%%%%%%%%%%%%%%%%%%%%%%%%%%%%%%%%%%%%%%%%%%%%%%%%%
\subsection{Thermal conductivity}
\begin{frame}{Cross-Plane Perfect SL}
\begin{figure}[t]
\begin{center}
\vspace*{-0.8cm}
\scalebox{0.75}{\includegraphics{KvL_cp.eps}}
\renewcommand{\figure}{Fig.}
\label{fig:cp}
\end{center}
\end{figure}
\end{frame}

%%%%%%%%%%%%%%%%%%%%%%%%%%%%%%%%%%%%%%%%%%%%%%%%%%%%%%
%%%%%%%%%%%%%%%%%%%%%%%%%%%%%%%%%%%%%%%%%%%%%%%%%%%%%%
\begin{frame}{Cross-Plane Mixed SL}
\begin{figure}[t]
\begin{center}
\vspace*{-0.8cm}
\scalebox{0.75}{\includegraphics{KvL_cp_all.eps}}
\renewcommand{\figure}{Fig.}
\label{fig:cp_all}
\end{center}
\end{figure}
\end{frame}

%%%%%%%%%%%%%%%%%%%%%%%%%%%%%%%%%%%%%%%%%%%%%%%%%%%%%%
%%%%%%%%%%%%%%%%%%%%%%%%%%%%%%%%%%%%%%%%%%%%%%%%%%%%%%
\begin{frame}{In-Plane Perfect SL}
\begin{figure}[t]
\begin{center}
\vspace*{-0.8cm}
\scalebox{0.75}{\includegraphics{KvL_ip.eps}}
\renewcommand{\figure}{Fig.}
\label{fig:ip}
\end{center}
\end{figure}
\end{frame}

%%%%%%%%%%%%%%%%%%%%%%%%%%%%%%%%%%%%%%%%%%%%%%%%%%%%%%
%%%%%%%%%%%%%%%%%%%%%%%%%%%%%%%%%%%%%%%%%%%%%%%%%%%%%%
\begin{frame}{In-Plane Mixed SL}
\begin{figure}[t]
\begin{center}
\vspace*{-0.8cm}
\scalebox{0.75}{\includegraphics{KvL_ip_all.eps}}
\renewcommand{\figure}{Fig.}
\label{fig:ip_all}
\end{center}
\end{figure}
\end{frame}

%%%%%%%%%%%%%%%%%%%%%%%%%%%%%%%%%%%%%%%%%%%%%%%%%%%%%%
%%%%%%%%%%%%%%%%%%%%%%%%%%%%%%%%%%%%%%%%%%%%%%%%%%%%%%
\section{Conclusions}
\begin{frame}{Contributions}
\begin{itemize}
\item Validated GK, DM, ALD, NMD methods for bulk
\item First to apply and validate NMD for superlattices
\item Developed a simple, parallel and open source workflow for NMD
\end{itemize}
\end{frame}

%%%%%%%%%%%%%%%%%%%%%%%%%%%%%%%%%%%%%%%%%%%%%%%%%%%%%%
%%%%%%%%%%%%%%%%%%%%%%%%%%%%%%%%%%%%%%%%%%%%%%%%%%%%%%
\begin{frame}{Conclusions}
\begin{itemize}
\item Bulk properties are not representative of short-period superlattices
\item Mixing \textit{breaks} secondary periodicity
\item Coherent effects = dispersion effects
\end{itemize}
\end{frame}

%%%%%%%%%%%%%%%%%%%%%%%%%%%%%%%%%%%%%%%%%%%%%%%%%%%%%%
%%%%%%%%%%%%%%%%%%%%%%%%%%%%%%%%%%%%%%%%%%%%%%%%%%%%%%
\end{document}
