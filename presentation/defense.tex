%%%%%%%%%%%%%%%%%%%%%%%%%%%%%%%%%%%%%%%%%%%%%%%%%%%%%%%%%%%%
%%  This Beamer template was created by Cameron Bracken.
%%  Anyone can freely use or modify it for any purpose
%%  without attribution.
%%
%%  Last Modified: January 9, 2009
%%

\documentclass[xcolor=x11names,compress]{beamer}

%% General document %%%%%%%%%%%%%%%%%%%%%%%%%%%%%%%%%%
\usepackage{graphicx}
\usepackage{epstopdf}
%\usepackage{tikz}
%%%%%%%%%%%%%%%%%%%%%%%%%%%%%%%%%%%%%%%%%%%%%%%%%%%%%%

%% Beamer Layout %%%%%%%%%%%%%%%%%%%%%%%%%%%%%%%%%%
\useoutertheme[subsection=false,shadow]{miniframes}
\useinnertheme{default}
\usefonttheme{serif}
\usepackage{palatino}

\setbeamersize{text margin left=5pt,text margin right=5pt}

\setbeamertemplate{navigation symbols}{}
\setbeamerfont{title like}{shape=\scshape}
\setbeamerfont{frametitle}{shape=\scshape}

\setbeamercolor*{lower separation line head}{bg=DeepSkyBlue4} 
\setbeamercolor*{normal text}{fg=black,bg=white} 
\setbeamercolor*{alerted text}{fg=red} 
\setbeamercolor*{example text}{fg=black} 
\setbeamercolor*{structure}{fg=black} 
 
\setbeamercolor*{palette tertiary}{fg=black,bg=black!10} 
\setbeamercolor*{palette quaternary}{fg=black,bg=black!10} 

\renewcommand{\(}{\begin{columns}}
\renewcommand{\)}{\end{columns}}
\newcommand{\<}[1]{\begin{column}{#1}}
\renewcommand{\>}{\end{column}}

\graphicspath{{/home/schuberm/Dropbox/git/plots.nogit/images/}}
%%%%%%%%%%%%%%%%%%%%%%%%%%%%%%%%%%%%%%%%%%%%%%%%%%


\begin{document}


%%%%%%%%%%%%%%%%%%%%%%%%%%%%%%%%%%%%%%%%%%%%%%%%%%%%%%
%%%%%%%%%%%%%%%%%%%%%%%%%%%%%%%%%%%%%%%%%%%%%%%%%%%%%%
\section{\scshape Introduction}
\begin{frame}
\title{Phonon Properties in Superlattices}
%\subtitle{SUBTITLE}
\author{
	Samuel Huberman\\
%	{\it Humboldt State University}\\
}
\date{
%	\begin{tikzpicture}[decoration=Koch curve type 2] 
%		\draw[DeepSkyBlue4] decorate{ decorate{ decorate{ (0,0) -- (3,0) }}}; 
%	\end{tikzpicture}  
	\\
	\vspace{1cm}
	\today
}
\titlepage
\end{frame}

%%%%%%%%%%%%%%%%%%%%%%%%%%%%%%%%%%%%%%%%%%%%%%%%%%%%%%
%%%%%%%%%%%%%%%%%%%%%%%%%%%%%%%%%%%%%%%%%%%%%%%%%%%%%%
\section{\scshape Background and Context}
\begin{frame}{Recent Papers}
\begin{itemize}
\item Garg et al. ``High Thermal Conductivity in Short-Period Superlattices''
\begin{itemize}
\item Noted the importance of interfacial disorder upon thermal conductivity predictions
\end{itemize}
\vspace{0.5cm}
\item Luckyanova, Garg et al. ``Coherent Phonon Heat Conduction in Superlattices'' 
\begin{itemize}
\item Reported linear increase in thermal conductivity with increasing number of layers
\end{itemize}
\end{itemize}
\end{frame}

%%%%%%%%%%%%%%%%%%%%%%%%%%%%%%%%%%%%%%%%%%%%%%%%%%%%%%
%%%%%%%%%%%%%%%%%%%%%%%%%%%%%%%%%%%%%%%%%%%%%%%%%%%%%%
\subsection{Superlattice Phonons}
\begin{frame}{Superlattice Phonons}
\begin{columns}
\column{.5\textwidth}
\textbf{Coherent Phonons}
\begin{itemize}
\item Satisfy Bragg reflection
\begin{equation*}\label{EQ:Coh}
2D \cos \theta = m \lambda
\end{equation*}
\item Brillouin oscillations
\item Impulse stimulate Raman scattering
\end{itemize}
Do \textit{NOT} transport heat
\column{.5\textwidth}
\textbf{Coherent Phonon Effects}
\begin{itemize}
\item Coherence length $(l_c)$ is defined as the distance a wave travels prior to the randomization of its phase
\item $\Lambda$ == $l_c$?
\end{itemize}
If a superlattice phonon has $l_c>D$, indicative of \textit{coherent transport}
\end{columns}
\end{frame}

%%%%%%%%%%%%%%%%%%%%%%%%%%%%%%%%%%%%%%%%%%%%%%%%%%%%%%
%%%%%%%%%%%%%%%%%%%%%%%%%%%%%%%%%%%%%%%%%%%%%%%%%%%%%%
\subsection{MD Domain}
\begin{frame}{\small{4x4}}
\begin{figure}[!h]
\vspace*{-0.5cm}
\begin{center}
\scalebox{0.2}{ \includegraphics{/home/schuberm/Dropbox/git/plots.nogit/images/4p_perfect.eps}}
\renewcommand{\figure}{Fig.}
\end{center}
\end{figure}
\begin{figure}[!h]
\vspace*{-0.25cm}
\begin{center}
\scalebox{0.2}{ \includegraphics{/home/schuberm/Dropbox/git/plots.nogit/images/4p_mixed.eps}}
\label{fig:mixed}
\end{center}
\end{figure}
\end{frame}



%%%%%%%%%%%%%%%%%%%%%%%%%%%%%%%%%%%%%%%%%%%%%%%%%%%%%%
%%%%%%%%%%%%%%%%%%%%%%%%%%%%%%%%%%%%%%%%%%%%%%%%%%%%%%
\section{\scshape Methodology}
\subsection{Dispersion}
\begin{frame}{\small{Dispersion}}
\begin{figure}[!h]
\vspace*{-0.5cm}
\begin{center}
\scalebox{0.45}{ \includegraphics{/home/schuberm/Dropbox/git/plots.nogit/images/dispersion.eps}}
\renewcommand{\figure}{Fig.}
\label{fig:sed}
\end{center}
\end{figure}
\begin{figure}[!h]
\vspace*{-0.25cm}
\begin{center}
\scalebox{0.45}{ \includegraphics{/home/schuberm/Dropbox/git/plots.nogit/images/dispersion_ip.eps}}
\renewcommand{\figure}{Fig.}
\label{fig:dispersion}
\end{center}
\end{figure}
\end{frame}


%%%%%%%%%%%%%%%%%%%%%%%%%%%%%%%%%%%%%%%%%%%%%%%%%%%%%%
%%%%%%%%%%%%%%%%%%%%%%%%%%%%%%%%%%%%%%%%%%%%%%%%%%%%%%
\subsection{SED}
\begin{frame}{SED}
\begin{columns}
\column{.075\textwidth}
\begin{figure}[t]
\vspace*{-1cm}
%\begin{center}
\includegraphics[height=70mm]{/home/schuberm/Dropbox/git/plots.nogit/images/4p_dispersion_only.eps}
\renewcommand{\figure}{Fig.}
\label{fig:disp_4p}
%\end{center}
\end{figure}
\column{.91\textwidth}
\begin{figure}[t]
%\begin{center}
\vspace*{-1cm}
\scalebox{0.60}{\includegraphics{/home/schuberm/Dropbox/git/plots.nogit/images/sed.eps}}
\renewcommand{\figure}{Fig.}
%\caption{(Colour online) Spectral energy density plots for the selected modes along k=[1,0,0] of a 4x4 superlattice as shown in Figure~\ref{fig:dispersion}. Dark blue corresponds to a superlattice without mixing, red corresponds to mixing of 80/20 and light blue corresponds to mixing of 60/40.}
\label{fig:sed}
%\end{center}
\end{figure}
\end{columns}
\end{frame}

%%%%%%%%%%%%%%%%%%%%%%%%%%%%%%%%%%%%%%%%%%%%%%%%%%%%%%
%%%%%%%%%%%%%%%%%%%%%%%%%%%%%%%%%%%%%%%%%%%%%%%%%%%%%%
\section{\scshape Results}
\subsection{Lifetimes}
\begin{frame}{Lifetimes}
\begin{figure}[!h]
\begin{center}
\scalebox{0.55}{\includegraphics{/home/schuberm/Dropbox/git/plots.nogit/images/lifvomega_pres.eps}}
\renewcommand{\figure}{Fig.}
\label{fig:lifetimes}
\end{center}
\end{figure}
\end{frame}

%%%%%%%%%%%%%%%%%%%%%%%%%%%%%%%%%%%%%%%%%%%%%%%%%%%%%%
%%%%%%%%%%%%%%%%%%%%%%%%%%%%%%%%%%%%%%%%%%%%%%%%%%%%%%
\subsection{MFP}
\begin{frame}{MFP}
\begin{figure}[!h]
\begin{center}
\scalebox{0.55}{\includegraphics{/home/schuberm/Dropbox/git/plots.nogit/images/MFP_cp_pres.eps}}
\renewcommand{\figure}{Fig.}
\label{fig:sed}
\end{center}
\end{figure}
\end{frame}

%%%%%%%%%%%%%%%%%%%%%%%%%%%%%%%%%%%%%%%%%%%%%%%%%%%%%%
%%%%%%%%%%%%%%%%%%%%%%%%%%%%%%%%%%%%%%%%%%%%%%%%%%%%%%
\begin{frame}{GK}

\begin{figure}%[H]
\begin{center}
\includegraphics[width=0.3\textwidth]{/home/schuberm/Dropbox/git/plots.nogit/images/GK_cp5.eps} \hspace{0.05\textwidth}%
\includegraphics[width=0.3\textwidth]{/home/schuberm/Dropbox/git/plots.nogit/images/GK_cp7.eps} \\[2em]
%\end{center}
%\end{figure}
%\column{.5\textwidth}
%\begin{figure}%[H]
%\begin{center}
\includegraphics[width=0.3\textwidth]{/home/schuberm/Dropbox/git/plots.nogit/images/GK_cp8.eps} \hspace{0.05\textwidth}%
\includegraphics[width=0.3\textwidth]{/home/schuberm/Dropbox/git/plots.nogit/images/GK_cp9.eps} \par
\end{center}
\end{figure}

\end{frame}

%%%%%%%%%%%%%%%%%%%%%%%%%%%%%%%%%%%%%%%%%%%%%%%%%%%%%%
%%%%%%%%%%%%%%%%%%%%%%%%%%%%%%%%%%%%%%%%%%%%%%%%%%%%%%
\begin{frame}{NMD vs GK}
\begin{figure}%[H]
\begin{center}
\scalebox{0.75}{\includegraphics{/home/schuberm/Dropbox/git/plots.nogit/images/KvL_24814.eps}}
\renewcommand{\figure}{Fig.}
\caption{Comparison between thermal conductivity predictions from GK and NMD.}
\label{FIG:NMD_v_GK}
\end{center}
\end{figure}
\end{frame}

%%%%%%%%%%%%%%%%%%%%%%%%%%%%%%%%%%%%%%%%%%%%%%%%%%%%%%
%%%%%%%%%%%%%%%%%%%%%%%%%%%%%%%%%%%%%%%%%%%%%%%%%%%%%%
\begin{frame}{Contributions}
\begin{itemize}
\item Validated GK,DM,ALD,NMD methods for bulk
\item Validated GK,ALD,NMD methods for superlattices
\item Bulk properties are not representative of short-period superlattices
\item Mixing \textit{breaks} secondary periodicity
\item Coherent effects = dispersion effects
\end{itemize}
\end{frame}

%%%%%%%%%%%%%%%%%%%%%%%%%%%%%%%%%%%%%%%%%%%%%%%%%%%%%%
%%%%%%%%%%%%%%%%%%%%%%%%%%%%%%%%%%%%%%%%%%%%%%%%%%%%%%
\end{document}
